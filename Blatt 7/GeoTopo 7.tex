\documentclass[a4paper,10pt]{article}
%\documentclass[a4paper,10pt]{scrartcl}

\usepackage{xltxtra}
\usepackage{../mystyle}

% die Nummer des Aufgabenblattes
\newcommand{\blattNummer}{7}

\setromanfont[Mapping=tex-text]{Linux Libertine O}
% \setsansfont[Mapping=tex-text]{DejaVu Sans}
% \setmonofont[Mapping=tex-text]{DejaVu Sans Mono}

\title{\sc Einführung in die Geometrie und Topologie \\ \Large Blatt 7}
\author{Jendrik Stelzner}
\date{\today}



\begin{document}
\maketitle





\section{}
Wir gehen davon aus, dass $X \neq \emptyset$, damit der Begriff der Fundamentalgruppe für $X$ Sinn ergibt.

\begin{figure}[ht]\centering
 \begin{tikzpicture}[auto, node distance = 6em]
  \node (R) {$(\R,r_0)$};
  \node (S1) [below of = R] {$(S^1,y_0)$};
  \node (X) [left of = S1, node distance = 8em] {$(X,x_0)$};
  \draw[->] (X) to node[swap] {$f_i$} (S1);
  \draw[->] (R) to node {$\exp$} (S1);
  \draw[->,dashed] (X) to node {$\tilde{f}_i$} (R);
 \end{tikzpicture}
 \caption{Liftung von $f_i$.}
 \label{fig: Liftung von f_i}
\end{figure}

Es genügt zu zeigen, dass je zwei stetige Abbildungen $f_1, f_2 : X \to S^1$ zueinander homotop sind. Es sei ein Basispunkt $x_0 \in X$ fixiert, $y_0 = f_1(x_0)$ und $r_0 \in \R$ mit $\exp(r_0) = y_0$. Es ist
\[
 {f_1}_* (\pi_1(X,x_0)) \subseteq \pi_1(S^1, y_0)
\]
eine endliche Untergruppe, wegen
\[
 \pi_1(S^1, y_0) \cong \Z
\]
also
\[
 f_1^*(\pi_1(X,x_0)) \cong 1
\]
und somit
\[
 {f_1}_*(\pi_1(X,x_0)) \subseteq \exp_*(\pi_1(\R,r_0)).
\]

Da $X$ zusammenhängend und lokal wegzusammenhängend ist, gibt es deshalb nach dem Liftungssatz eine Liftung $\tilde{f}_1 : (X,x_0) \to (\R, r_0)$ von $f_1$, siehe Abbildung \ref{fig: Liftung von f_i}. Analog gibt es auch eine Liftung $\tilde{f}_2 : X \to \R$ von $f_2$. Da $\R$ zusammenziehbar ist, ist $\tilde{f}_1 \simeq \tilde{f}_2$ homotop. Daher ist auch
\[
 f_1 = \exp \tilde{f}_1 \simeq f_2 = \exp \tilde{f}_2.
\]





\section{}
Es ist klar, dass
\[
 (X \times \R) \times \Z \to X \times \R, ((x,t),n) \mapsto \left( f^{n}(x), t-n \right).
\]
eine Gruppenwirkung von $\Z$ auf $X \times \R$ definiert. Für alle $n \in \Z$ ist die Abbildung
\[
 X \times \R \to X \times \R, (x,t) \mapsto (x,t) \cdot n =  \left( f^{n}(x), t-n \right)
\]
stetig, da sie in jeder Koordinate stetig ist. Die Gruppenwirkung ist eigentlich diskontinuierlich, da für alle $(x,t) \in X \times \R$ für die Umgebung
\[
 U := X \times \left( t-\frac{1}{3} , t + \frac{1}{3} \right) \subseteq X \times \R
\]
von $(x,t)$ für alle $n \in \Z, n \neq 0$
\[
 Un \cap U
 = X \times \left( t-\frac{1}{3}-n, t+\frac{1}{3}-n \right) \cap X \times \left( t-\frac{1}{3}, t+\frac{1}{3} \right)
 = \emptyset.
\]
Es bezeichne $\sim_\Z$ die durch die Gruppenwirkung induzierte Äquivalenzrelation auf $X \times [0,1]$ und der Homöomorphismus $g : X \times \R \to X \times \R$ definiert als
\[
 g : X \times \R \to X \times \R, (x,t) \mapsto (x,t) \cdot 1 = \left( f(x), t-1 \right).
\]

Es bezeichne
\[
 \iota : X \times [0,1] \to X \times \R
\]
die kanonische Inklusion. Weiter seien
\[
 p : X \times [0,1] \to T_f
\]
und
\[
 \pi : X \times \R \to (X \times \R)/\Z
\]
die kanonischen Projektionen. Es ist klar, dass $\sim$ die Einschränkung von $\sim_\Z$ auf $X \times [0,1]$ ist, und dass deshalb die stetige Funktion
\[
 \pi \iota : X \times [0,1] \to (X \times \R)/\Z
\]
als Bijektion
\[
 \varphi : T_f \to (X \times \R)/\Z
\]
faktorisiert. Diese ist nach der universellen Eigenschaft der Quotientenraumtopologie ebenfalls stetig. Wir erhalten daher das kommutative Diagramm in Abbildung \ref{fig: Abbildungstorus}.
\begin{figure}[ht]\centering
 \begin{tikzpicture}[auto, node distance = 6em]
  \node (X x I) {$X \times [0,1]$};
  \node (X x R) [right of = X x I, node distance = 8em] {$X \times \R$};
  \node (Tf) [below of = X x I] {$T_f$};
  \node (X x R / Z) [below of = X x R] {$(X \times \R)/\Z$};
  \draw[->] (X x I) to node {$\iota$} (X x R);
  \draw[->] (Tf) to node[swap] {$\varphi$} (X x R / Z);
  \draw[->] (X x I) to node[swap] {$p$} (Tf);
  \draw[->] (X x R) to node {$\pi$} (X x R / Z);
 \end{tikzpicture}
 \caption{Homöomorphie zum Abbildungstorus.}
 \label{fig: Abbildungstorus}
\end{figure}

Wir zeigen, dass $\varphi$ auch offen ist. Hierfür sei $U \subseteq T_f$ offen und nichtleer. Wir setzen
\[
 V := p^{-1}(U) \text{ und }
 W := \pi^{-1}(\varphi(U)).
\]
Es ist klar, dass $V$ offen in $X \times [0,1]$ ist, und dass $W$ die Saturierung von $V$ bezüglich $\sim_\Z$ in $X \times \R$ ist. Außerdem ist die Offenheit von $\varphi(U)$ nach der Definition der Quotientenraumtopologie äquivalent zu der Offenheit von $W$.

Es sei $(x,t) \in W$ beliebig aber fest.

Ist $t \not\in \Z$, so gibt es $(y,u) \in V$ mit $(x,t) \sim_\Z (y,u)$, also
\[
 (x,t) = (f^n(y),u-n) = g^n(y,u)
\]
für passendes $n \in \N$. Da $V$ offen in $X \times [0,1]$ ist, gibt es eine offene Umgebung $O \subseteq X$ von $y$ und ein offenes Intervall $(a,b) \subseteq [0,1]$ mit $u \in (a,b)$, so dass
\[
 (y,u) \in O \times (a,b) \subseteq V.
\]
Daher ist
\[
 (x,t) = g^n(y,u) \subseteq g^n(O \times (a,b)) = f^n(O) \times (a-n,b-n),
\]
wobei $f^n(O) \times (a-n,b-n)$ offen ist, da $f$ ein Homöomorphismus ist.

Ist $t \in \Z$, so ist, da $W$ die Saturierung von $V$ bezüglich $\sim_\Z$ ist, und $V$ bezüglich $\sim$ saturiert ist
\[
 \left( f^{t-1}(x) , 1 \right) = (x,t) \cdot (t-1) \in V \text{ und }
 \left( f^{t}(x) , 0 \right) = (x,t) \cdot t \in V.
\]
Da $V$ in $X \times [0,1]$ offen ist, gibt es daher eine offene Umgebung $O$ von $f^{t-1}(x)$ und $a \in (0,1)$ mit
\[
 \left( f^{t-1}(x), 1 \right) \in O \times (a,1] \subseteq V \subseteq W,
\]
und eine offene Umgebung $O'$ von $f^{t}(x)$ und $a \in (0,1)$ mit
\[
 \left( f^{t}(x), 0 \right) \in O' \times [0,b) \subseteq V \subseteq W.
\]
Da $W$ bezüglich $\sim_\Z$ saturiert ist, ist daher auch
\[
 g\left( O \times (a,1] \right) = f(O) \times (a-1,0] \subseteq W,
\]
wobei
\[
 \left( f^{t}(x), 0 \right) = g\left( f^{t-1}(x), 1 \right) \subseteq f(O) \times (a-1,0].
\]
Es ist deshalb
\[
 \tilde{O} := (f(O) \cap O') \times (a-1,b) \in W
\]
eine offene Umgebung von $\left( f^{t}(x), 0 \right)$. Daher ist $g^{-t}(\tilde{O}) \subseteq W$ eine offene Umgebung von $g^{-t}(f^t(x),0) = (x,t)$.

Da die Menge $W$ um jeden ihrerer Punkte einen offene Umgebung enthält, ist $W$ offen. Wegen der Beliebigkeit von $U$ zeigt dies, dass $\varphi$ offen ist. Also ist $\varphi$ ein Homöomorphismus.





\section{}


\subsection{}
Die Multiplikation ist assoziativ, da für alle $(n, m), (n', m'), (n'', m'') \in \Z \times \Z$
\begin{align*}
  &\, ((n, m) \cdot (n', m')) \cdot (n'', m'') \\
 =&\, \left(n + (-1)^m n', m + m'\right) \cdot (n'', m'') \\
 =&\, \left(n + (-1)^m n' + (-1)^{m+m'} n'', m + m' + m''\right) \\
 =&\, \left(n + (-1)^m \left(n' + (-1)^{m'} n''\right) ,m + m' + m''\right) \\
 =&\, (n, m) \cdot (n' + (-1)^{m'} n'', m' + m'') \\
 =&\, (n, m) \cdot ((n', m') \cdot (n'', m'')).
\end{align*}
Das Element $(0,0) \in \Z \times \Z$ ist bezüglich der Multiplikation rechtsneutral, da für alle $(n,m) \in \Z \times \Z$
\[
 (n,m) \cdot (0,0) = (n + (-1)^0 \cdot 0, m + 0) = (n,m).
\]
Das Element $(n,m) \in \Z \times \Z$ hat das Rechtsinverse $((-1)^{m+1}n,-m) \in \Z \times \Z$, da
\[
 (n,m) \cdot \left((-1)^{m+1} n, -m\right)
 = (n + (-1)^m (-1)^{m+1} n, m - m)
 = (0, 0).
\]
Das zeigt, dass $\Z \rtimes \Z$ eine Gruppe ist. Sie ist nicht abelsch, da etwa
\[
 (1,0) \cdot (1,1) = (2,1) \neq (0,1) = (1,1) \cdot (1,0).
\]


\subsection{}
Es handelt sich um eine Gruppenwirkung von $\Z \rtimes \Z$ auf $\R^2$, da für alle $(x,y) \in \R^2$
\[
 (x,y) \cdot (0,0) = (x,y),
\]
und für alle $(n,m), (n', m') \in \Z \rtimes \Z$ und $(x,y) \in \R^2$
\begin{align*}
  &\, ((x,y) \cdot (n,m)) \cdot (n',m') \\
 =&\, ((-1)^m (x + n), y + m) \cdot (n',m') \\
 =&\, \left( (-1)^{m'}((-1)^m (x + n) + n'), y + m + m' \right) \\
 =&\, \left( (-1)^{m+m'} (x + n + (-1)^m n'), y + m + m' \right) \\
 =&\, (x,y) \cdot (n + (-1)^m n', m + m') \\
 =&\, (x,y) \cdot ((n,m) \cdot (n',m')).
\end{align*}

Die Stetigkeit der Gruppenwirkung ist klar, da sie in jeder Komponente stetig ist. Die Gruppenwirkung ist eigentlich diskontinuierlich, denn für alle $(x,y) \in \R^2$ ist 
\[
 U = \left( x-\frac{1}{3}, x+\frac{1}{3} \right) \times \left( y-\frac{1}{3}, y+\frac{1}{3} \right)
\]
eine Umgebung von $(x,y)$, für die für alle $(n,m) \in \Z \rtimes \Z$ mit $(n,m) \neq (0,0)$
\[
 U \cdot (n,m) \cap U = \emptyset.
\]





\section{}
Es ist klar, dass die Inklusion
\[
 j : X \to X \times \R, x \mapsto (x,0),
\]
stetig ist. Aus Aufgabe 7.2 ist bekant, dass die Abbildung
\[
 p : X \times \R \to T_f, (x,t) \mapsto [(x,t)].
\]
eine Überlagerungsabbildung ist. Da offenbar $\iota = pj$ ist $\iota$ ebenfalls stetig.

Bezeichnen
\[
 \pi_1 : X \times \R \to X \text{ und } \pi_2 : X \times \R \to \R
\]
die kanonischen Projektionen, so ist es klar, dass die Abbildung
\[
 \exp \pi_2 : X \times \R \to S^1
\]
über $T_f$ faktorisiert, wobei es sich bei der faktorisierten Abbildung offenbar um $q$ handelt. Inbesondere ist $q$ daher wohldefiniert. Aus der universellen Eigenschaft des Quotientenraumtopologie folgt weiter, dass $q$ stetig ist. Insgesamt erhalten wir damit das kommutative Diagramm in Abbildung \ref{fig: großes Diagramm}.
\begin{figure}\centering
 \begin{tikzpicture}[auto]
  \node (X x R) {$X \times \R$};
  \node (Tf) [below of = X x R, node distance = 6em] {$T_f$};
  \node (X)  [left of = Tf, node distance = 8em] {$X$};
  \node (S1) [right of = Tf, node distance = 8em] {$S^1$};
  \draw[->] (X) to node[swap] {$\iota$} (Tf);
  \draw[->] (Tf) to node[swap] {$q$} (S1);
  \draw[->] (X x R) to node {$p$} (Tf);
  \draw[->] (X) to node {$j$} (X x R);
  \draw[->] (X x R) to node {$\exp \pi_2$} (S1);
 \end{tikzpicture}
 \caption{Ein Diagramm.}
 \label{fig: großes Diagramm}
\end{figure}


\subsection{}
Es sei $x \in X$ beliebig aber fest. Da
\[
 \iota_* = (pj)_* = p_* j_*
\]
genügt es für die Injektivität von $\iota_*$ zu zeigen, dass
\[
 j_* : \pi_1(X,x) \to \pi_1(X \times \R, (x,0))
\]
und
\[
 p_* : \pi_1(X \times \R, (x,0)) \to \pi_1(T_f, \iota(x))
\]
injektiv sind. Da $p$ eine Überlagerungsabbildung ist, ist $p_*$ bekanntermaßen injektiv.

Es sei $\gamma : (S^1,1) \to (X,x)$ mit $[\gamma] \in \ker j_*$. Dann ist
\[
 j \gamma : (S^1,1) \to (X \times \R, (x,0))
\]
homotop zur konstanten Schleife relativ zu $1$. Es gibt also eine Homotopie
\[
 F : S^1 \times I \to X \times \R,
\]
so dass
\[
 F(z,0) = j\gamma(z) \text{ und } F(z,1) = (x,0) \text{ für alle } z \in S^1
\]
und
\[
 F(0,t) = (x,0) \text{ für alle } t \in I.
\]
Es ist dann
\[
 \pi_1 F : S^1 \times I \to X
\]
eine Homotopie von $\gamma$ zur konstanten Schleife relativ zu $1$. Daher ist $[\gamma]$ das neutrale Element. Das zeigt, dass $\ker j_*$ trivial ist, also $j_*$ injektiv.


\subsection{}
Es sei $[(x,t)] \in T_f$ beliebig aber fest.

Da $x$ wegzusammenhängend ist, gibt es einen Weg $\gamma : [t-1,t] \to X$ von $f(x)$ nach $x$. Es ist daher
\[
 \tilde{\gamma} : [t-1,t] \to X \times \R, s \mapsto (\gamma(s),s)
\]
ein Weg von $(f(x),t-1)$ nach $(x,t)$. Daher ist
\[
 p \tilde{\gamma} : [t-1,t] \to T_f, s \mapsto [(\gamma(s),s)]
\]
eine Schlinge mit
\[
 p\tilde{\gamma}(0) = p\tilde{\gamma}(1) = [(x,t)].
\]
Über die Abbildung
\[
 \exp : [t-1,t] \to S^1
\]
faktorisiert $p\tilde{\gamma}$ als eindeutige, nach der universellen Eigenschaft der Quotientenraumtopologie stetige, Abbildung
\[
 s : S^1 \to T_f,
\]
d.h. $p \tilde{\gamma} = s \exp$. Wir bemerken, dass auch
\[
 q p \tilde{\gamma} = \exp.
\]
Wir erhalten also die beiden kommutativen Diagramme in Abbildung TODO.

Da
\[
 q s \exp = q p \tilde{\gamma} = \exp
\]
folgt aus der Surjektivität von $\exp : [t-1,t] \to S^1$, dass $qs = \id_{S^1}$. Da
\[
 q : (T_f, [(x,t)]) \to (S^1, \exp(t))
\]
und
\[
 s : (S^1, \exp(t)) = (T_f, [(x,t)])
\]
ist deshalb
\[
 q_* s_* = (qs)_* = {\id_{S^1}}_* = \id_{\pi_1(S^1,\exp(t))}.
\]
Wegen der Beliebigkeit von $[(x,t)] \in T_f$ zeigt dies, dass $q_*$ split-surjektiv ist.


\subsection{}
Es sei $x \in X$ beliebig aber fest.

Es sei zunächst $[\tilde{\gamma}] \in \im \iota_*$. Dann gibt es $\gamma : (S^1, 1) \to (X,x)$ mit $\tilde{\gamma} = \iota \gamma$. Da für alle $z \in S^1$
\[
 q \tilde{\gamma}(z)
 = q \iota \gamma(z)
 = q ([(\gamma(z),0)])
 = \exp(0)
 = 1
\]
ist $q_*([\tilde{\gamma}])$ das neutrale Element, also $[\tilde{\gamma}] \in \ker q_*$. Das zeigt, dass $\im \iota_* \subseteq \ker q_*$.

Sei andererseits $[\tilde{\gamma}] \in \ker q_*$. Da $p$ ein Überlagerungsabbildung ist, lässt sich die Schlinge $\tilde{\gamma} : I \to T_f$ nach dem Wege-Liftungssatz zu einem Weg $\gamma : X \times I \to S^1$ mit Anfangpunkt $(x,0)$ liften. Da
\[
 p \gamma(0) = \tilde{\gamma}(0) = \tilde{\gamma}(1) = p \gamma (1)
\]
ist $p \gamma(1) = (x,0)$. Da die Schlinge
\[
 q \tilde{\gamma}
 = q p \gamma
 = \exp \pi_2 \gamma
\]
nullhomotop ist, muss sie Windungszahl $0$ haben. Da die Windungszahl ist gerade $\pi_2 \gamma(0) - \pi_2 \gamma(1)$ ist, muss also
\[
 \pi_2 \gamma(0) = \pi_2 \gamma(1).
\]
Da $\gamma(0)$ und $\gamma(1)$ in der zweiten Koordinate übereinstimmen, und $p \gamma(0) = p \gamma(1)$ müssen sie auch in der ersten Koordinate übereinstimmen. (Da für festes $t \in \R$ in $X \times \{t\}$ keine Punkte miteinander identifiziert werden.) Also ist $\gamma(0) = \gamma(1)$ und $\gamma$ somit ebenfalls eine Schlinge.

Wir betrachten die Homotopie
\[
 F : [0,1] \times I \to X \times \R, (t,s) \mapsto ( \pi_1(\gamma(t)), s \pi_2(\gamma(t)) ).
\]
Da für alle $t \in [0,1]$
\[
 F(t,0) = (\pi_1(\gamma(t)),0) \text{ und }
 F(t,1) = \gamma(t),
\]
sowie für alle $s \in [0,1]$
\[
 F(0,s) = F(1,s) = (x,0)
\]
ist $F$ eine Homotopie von $\gamma$ zu einer Schlinge $\tau$, die komplett in $X \times \{0\}$ liegt, relativ zum Anfangs- und Endpunkt. Es ist daher
\[
 p F : [0,1] \times I \to T_f
\]
eine Homotopie von $\tilde{\gamma}$ zu einer Schlinge $\tilde{\tau}$, die komplett in $p(X \times \{0\}) = \im \iota$ liegt, relativ zum Anfangs-und Endpunkt. Daher ist
\[
 [\tilde{\gamma}] = [\tilde{\tau}].
\]
Da $\tau$ komplett in $X \times \{0\}$ verläuft ist auch
\[
 \tilde{\tau} = p \tau = \iota \pi_1 \tau,
\]
also
\[
 [\tilde{\tau}] = \iota_* [\pi_1 \tau] \in \im \iota_*.
\]
Dies zeigt, dass $\ker q_* \subseteq \im \iota_*$.














































\end{document} 
