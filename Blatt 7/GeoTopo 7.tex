\documentclass[a4paper,10pt]{article}
%\documentclass[a4paper,10pt]{scrartcl}

\usepackage{xltxtra}
\usepackage{../mystyle}

% die Nummer des Aufgabenblattes
\newcommand{\blattNummer}{7}

\setromanfont[Mapping=tex-text]{Linux Libertine O}
% \setsansfont[Mapping=tex-text]{DejaVu Sans}
% \setmonofont[Mapping=tex-text]{DejaVu Sans Mono}

\title{\sc Einführung in die Geometrie und Topologie \\ \Large Blatt 7}
\author{Jendrik Stelzner}
\date{\today}



\begin{document}
\maketitle





\addtocounter{section}{2}





\section{}


\subsection{}
Die Multiplikation ist assoziativ, da für alle $(n, m), (n', m'), (n'', m'') \in \Z \times \Z$
\begin{align*}
  &\, ((n, m) \cdot (n', m')) \cdot (n'', m'') \\
 =&\, \left(n + (-1)^m n', m + m'\right) \cdot (n'', m'') \\
 =&\, \left(n + (-1)^m n' + (-1)^{m+m'} n'', m + m' + m''\right) \\
 =&\, \left(n + (-1)^m \left(n' + (-1)^{m'} n''\right) ,m + m' + m''\right) \\
 =&\, (n, m) \cdot (n' + (-1)^{m'} n'', m' + m'') \\
 =&\, (n, m) \cdot ((n', m') \cdot (n'', m'')).
\end{align*}
Das Element $(0,0) \in \Z \times \Z$ ist bezüglich der Multiplikation rechtsneutral, da für alle $(n,m) \in \Z \times \Z$
\[
 (n,m) \cdot (0,0) = (n + (-1)^0 \cdot 0, m + 0) = (n,m).
\]
Das Element $(n,m) \in \Z \times \Z$ hat das Rechtsinverse $((-1)^{m+1}n,-m) \in \Z \times \Z$, da
\[
 (n,m) \cdot \left((-1)^{m+1} n, -m\right)
 = (n + (-1)^m (-1)^{m+1} n, m - m)
 = (0, 0).
\]
Das zeigt, dass $\Z \rtimes \Z$ eine Gruppe ist. Sie ist nicht abelsch, da etwa
\[
 (1,0) \cdot (1,1) = (2,1) \neq (0,1) = (1,1) \cdot (1,0).
\]


\subsection{}
Es handelt sich um eine Gruppenwirkung von $\Z \rtimes \Z$ auf $\R^2$, da für alle $(x,y) \in \R^2$
\[
 (x,y) \cdot (0,0) = (x,y),
\]
und für alle $(n,m), (n', m') \in \Z \rtimes \Z$ und $(x,y) \in \R^2$
\begin{align*}
  &\, ((x,y) \cdot (n,m)) \cdot (n',m') \\
 =&\, ((-1)^m (x + n), y + m) \cdot (n',m') \\
 =&\, \left( (-1)^{m'}((-1)^m (x + n) + n'), y + m + m' \right) \\
 =&\, \left( (-1)^{m+m'} (x + n + (-1)^m n'), y + m + m' \right) \\
 =&\, (x,y) \cdot (n + (-1)^m n', m + m') \\
 =&\, (x,y) \cdot ((n,m) \cdot (n',m')).
\end{align*}

Die Stetigkeit der Gruppenwirkung ist klar, da sie in jeder Komponente stetig ist. Die Gruppenwirkung ist eigentlich diskontinuierlich, denn für alle $(x,y) \in \R^2$ ist 
\[
 U = \left( x-\frac{1}{3}, x+\frac{1}{3} \right) \times \left( y-\frac{1}{3}, y+\frac{1}{3} \right)
\]
eine Umgebung von $(x,y)$, für die für alle $(n,m) \in \Z \rtimes \Z$ mit $(n,m) \neq (0,0)$
\[
 U \cdot (n,m) \cap U = \emptyset.
\]


















\end{document} 
