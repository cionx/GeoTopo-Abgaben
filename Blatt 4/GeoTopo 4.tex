\documentclass[a4paper,10pt]{article}
%\documentclass[a4paper,10pt]{scrartcl}

\usepackage{xltxtra}
\usepackage{../mystyle}

% die Nummer des Aufgabenblattes
\newcommand{\blattNummer}{4}

\setromanfont[Mapping=tex-text]{Linux Libertine O}
% \setsansfont[Mapping=tex-text]{DejaVu Sans}
% \setmonofont[Mapping=tex-text]{DejaVu Sans Mono}

\title{\sc Einführung in die Geometrie und Topologie \\ \Large Blatt 4}
\author{Jendrik Stelzner}
\date{\today}

\begin{document}
\maketitle





\section{}


\subsection{}
Es sei $W \subseteq X \times Y$ offen und beliebig aber fest. Da die Mengen der Form $U \times V \subseteq X \times Y$ mit $U \subseteq X$ offen und $V \subseteq Y$ offen eine topologische Basis von $X \times Y$ bilden, gibt es offene Mengen  $\{U_i | i \in I\} \subseteq X$ und $\{V_i | i \in I\} \subseteq Y$ mit
\[
 W = \bigcup_{i \in I} (U_i \times V_i).
\]
Daher sind
\[
 p_1(W) = \bigcup_{i \in I} U_i \subseteq X \text{ und } p_2(W) = \bigcup_{i \in I} V_i \subseteq Y
\]
in den jeweiligen Räumen offen. Wegen der Beliebigkeit von $W$ folgt, dass $p_1$ und $p_2$ offen sind.












\end{document}
