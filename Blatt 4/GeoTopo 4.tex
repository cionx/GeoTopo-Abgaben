\documentclass[a4paper,10pt]{article}
%\documentclass[a4paper,10pt]{scrartcl}

\usepackage{xltxtra}
\usepackage{../mystyle}

% die Nummer des Aufgabenblattes
\newcommand{\blattNummer}{4}

\setromanfont[Mapping=tex-text]{Linux Libertine O}
% \setsansfont[Mapping=tex-text]{DejaVu Sans}
% \setmonofont[Mapping=tex-text]{DejaVu Sans Mono}

\title{\sc Einführung in die Geometrie und Topologie \\ \Large Blatt 4}
\author{Jendrik Stelzner}
\date{\today}



\begin{document}
\maketitle





\section{}


\subsection{}
Es sei $W \subseteq X \times Y$ offen und beliebig aber fest. Da die Mengen der Form $U \times V \subseteq X \times Y$ mit $U \subseteq X$ offen und $V \subseteq Y$ offen eine topologische Basis von $X \times Y$ bilden, gibt es offene Mengen  $\{U_i | i \in I\} \subseteq X$ und $\{V_i | i \in I\} \subseteq Y$ mit
\[
 W = \bigcup_{i \in I} (U_i \times V_i).
\]
Daher sind
\[
 p_1(W) = \bigcup_{i \in I} U_i \subseteq X \text{ und } p_2(W) = \bigcup_{i \in I} V_i \subseteq Y
\]
in den jeweiligen Räumen offen. Wegen der Beliebigkeit von $W$ folgt, dass $p_1$ und $p_2$ offen sind.





\addtocounter{section}{1}





\section{}

Für alle $j \in I$ bezeichnen wir die kanonische Projektion $\prod_{i \in I} X_i \to X_j$ mit $\pi_j$, und für alle $i \in I$ setzen wir $f_i := f \circ \pi_i$. Ist $f$ stetig, so ist $f_i$ als Verknüpfung stetiger Funktionen für alle $i \in I$ stetig.

Angenommen $f_i$ ist für alle $i \in I$ stetig. Für paarweise verschiedene Indizes $i_1, \ldots, i_n \in I$ und beliebige offene Mengen $U_1 \in X_{i_1}, \ldots, U_n \in X_{i_n}$ setzen wir
\[
 P_{i_1, \ldots, i_n}^{U_1, \ldots, U_n}
 = \prod_{i \in I} \begin{cases} U_k & \text{falls } i = i_k, \\ X_i & \text{sonst}, \end{cases}
 \subseteq \prod_{i \in I} X_i.
\]
Da die Mengen dieser Form eine topologische Basis von $\prod_{i \in I} X_i$ bilden, genügt es zum Nachweis der Stetigkeit von $f$ zu zeigen, dass
\[
 f^{-1}\left(P_{i_1, \ldots, i_n}^{U_1, \ldots, U_n}\right) \subseteq T
\]
offen ist für alle paarweise verschiedenen Indizes $i_1, \ldots, i_n \in I$ und beliebige offene Mengen $U_1 \in X_{i_1}, \ldots, U_n \in X_{i_n}$.

Seien also $i_1, \ldots, i_n$ und $U_1, \ldots, U_n$ wie zuvor beliebig aber fest. Wir bemerken, dass
\[
 P_{i_1, \ldots, i_n}^{U_1, \ldots, U_n} = \bigcap_{k=1}^n \pi_{i_k}^{-1}(U_k),
\]
und deshalb
\begin{align*}
 f^{-1}\left(P_{i_1, \ldots, i_n}^{U_1, \ldots, U_n}\right)
 &= f^{-1}\left(\bigcap_{k=1}^n \pi_{i_k}^{-1}(U_k)\right)
 = \bigcap_{k=1}^n (\pi_{i_k} \circ f)^{-1}(U_k)
 &= \bigcap_{k=1}^n f_{i_k}^{-1}(U_k).
\end{align*}
Da $U_k$ für alle $1 \leq k \leq n$ offen ist, und die $f_i$ alle stetig sind, ist $f_{i_k}^{-1}(U_k)$ für alle $1 \leq k \leq n$ offen, also $f^{-1}(P_{i_1, \ldots, i_n}^{U_1, \ldots, U_n})$ als endlicher Schnitt offener Mengen offen.

Wegen der Beliebigkeit von $i_1, \ldots, i_n$ und $U_1, \ldots, U_n$ zeigt dies die Stetigkeit von $f$.












\end{document} 
