\documentclass[a4paper,10pt]{article}
%\documentclass[a4paper,10pt]{scrartcl}

\usepackage{xltxtra}
\usepackage{../mystyle}

% die Nummer des Aufgabenblattes
\newcommand{\blattNummer}{4}

\setromanfont[Mapping=tex-text]{Linux Libertine O}
% \setsansfont[Mapping=tex-text]{DejaVu Sans}
% \setmonofont[Mapping=tex-text]{DejaVu Sans Mono}

\title{\sc Einführung in die Geometrie und Topologie \\ \Large Blatt 4}
\author{Jendrik Stelzner}
\date{\today}



\begin{document}
\maketitle


\begin{lem}[Universelle Eigenschaft des Koproduktes]\label{lem: universell Koprodukt}
 Es sei $I$ eine nichtleere Menge und $(X_i)_{i \in I}$ eine Familie topologischer Räume. Dann besitzt das Koprodukt $\coprod_{i \in I} X_i$ die folgende universelle Eigenschaft: Bezeichnet für alle $j \in I$
 \[
  \iota_j : X_j \to \coprod_{i \in I} X_i, x \mapsto (x,j)
 \]
 die kanonische Inklusion, so gibt es für jeden topologischen Raum $Y$ und jede Familie von Abbildungen $(f_i)_{i \in I}$ mit
 \[
  f_j : X_j \to Y \text{ für alle } j \in I
 \]
 eine eindeutige Abbildung $f : \coprod_{i \in I} X_i \to Y$, so dass $f_j = f \circ \iota_j$ für alle $j \in I$, also das Diagramm
 \begin{center}
  \begin{tikzpicture}[auto]
   \node (X_j) {$X_j$};
   \node [below of = X_j, node distance = 6em] (coprod X_i) {$\coprod_{i \in I} X_i$};
   \node [right of = coprod X_i, node distance = 8em] (Y) {$Y$};
   \draw[->] (X_j) to node[swap]{$\iota_j$} (coprod X_i);
   \draw[->] (X_j) to node{$f_j$} (Y);
   \draw[->] (coprod X_i) to node[swap]{$f$} (Y);
  \end{tikzpicture}
 \end{center}
für alle $j \in I$ kommutiert. $f$ ist genau dann stetig, wenn $f_j$ für alle $j \in I$ stetig ist.
\end{lem}
\begin{proof}
 Dass $f_j = f \circ \iota_j$ für alle $j \in I$ ist offenbar äquivalent dazu, dass für alle $(x,j) \in \coprod_{i \in I} X_i$
 \[
  f((x,j)) = f(\iota_j(x)) = f_j(x).
 \]
 Definiert man $f$ hierdurch, so ist $f$ offenbar wohldefiniert. Das zeigt die Existenz. Die Eindeutigkeit ist klar, da jedes Element $y \in \coprod_{i \in I} X_i$ von der Form $y = (x,j)$ mit $j \in I$ und $x \in X_j$ ist.
 
 Ist $f$ stetig, so ist $f_j = f \circ \iota_j$ für alle $j \in I$ als Verknüpfung stetiger Abbildungen ebenfalls stetig, da die Inklusion $\iota_j$ für alle $j \in I$ stetig ist.
 
 Ist $f_j$ für alle $j \in I$ stetig, so ergibt sich die Stetigkeit von $f$ wie folgt: Es sei $U \subseteq Y$ offen, beliebig aber fest. Da $f_j$ für alle $j \in I$ stetig ist, ist $f_j^{-1}(U) \subseteq X_j$ für alle $j \in I$ offen. Nach der Definition des Koproduktes ist eine Teilmenge $V \subseteq \coprod_{i \in I} X_i$ genau offen ist, wenn $\iota_j^{-1}(V) \subseteq X_j$ für alle $j \in I$ offen ist. Da
 \[
  \iota_j^{-1}(f^{-1}(U))
  = (f \circ \iota_j)^{-1}(U)
  = f_j^{-1}(U)
 \]
 für alle $j \in I$ offen ist, ist daher $f^{-1}(U) \subseteq \coprod_{i \in I} X_i$ offen. Wegen der Beliebigkeit von $U$ zeigt dies die Stetigkeit von $f$.
\end{proof}





\section{}


\subsection{}
Es sei $W \subseteq X \times Y$ offen und beliebig aber fest. Da die Mengen der Form $U \times V \subseteq X \times Y$ mit $U \subseteq X$ offen und $V \subseteq Y$ offen eine topologische Basis von $X \times Y$ bilden, gibt es offene Mengen  $\{U_i | i \in I\} \subseteq X$ und $\{V_i | i \in I\} \subseteq Y$ mit
\[
 W = \bigcup_{i \in I} (U_i \times V_i).
\]
Daher sind
\[
 p_1(W) = \bigcup_{i \in I} U_i \subseteq X \text{ und } p_2(W) = \bigcup_{i \in I} V_i \subseteq Y
\]
in den jeweiligen Räumen offen. Wegen der Beliebigkeit von $W$ folgt, dass $p_1$ und $p_2$ offen sind.


\subsection{}
Bekanntermaßen ist $[0,1) \times [0,1) \cong [0,1)^2$, wobei $p_1$ und $p_2$ den kanonischen Projektionen $\pi_1, \pi_2 : [0,1)^2 \to [0,1)$ entsprechen. Diese sind nicht abgeschlossen: Es ist $\overline{B_1((1,1))} \subseteq \R^2$ abgeschlossen, also auch
\[
 C := [0,1)^2 \cap \overline{B_1((1,1))} \subseteq [0,1)^2.
\]
Es sind aber $\pi_1(C) = \pi_2 = (0,1)$ nicht abgeschlossen in $[0,1)$.


\subsection{}
Da $Y \neq \emptyset$ ist $p_1$ surjektiv.

Es ist daher $U \subseteq X$ genau dann offen, wenn $p_1^{-1}(U) = U \times Y \subseteq X \times Y$ offen ist: Ist $U$ offen, so ist es klar, dass auch $U \times Y$ offen ist. Ist andererseits $U \times Y$ offen, so ist wegen der Offenheit und Surjektivität von $p_1$ auch
\[
 U = p_1(p_1^{-1}(U)) = p_1(U \times Y)
\]
offen.
Da $U \subseteq X$ genau dann offen ist, wenn $p_1^{-1}(U)$ offen ist, ist $p_1$ eine Quotientenraumabbildung.





\section{}

\begin{lem}\label{lem: Funktionsprodukt}
 Seien $X_1, X_2, T_1, T_2$ topologische Räume, $f_1 : X_1 \to T_1$ und $f_2 : X_2 \to T_2$ stetige Abbildungen. Dann ist auch die Abbildung
 \[
  f_1 \times f_2 : X_1 \times X_2 \to T_1 \times T_2, (x_1, x_2) \mapsto (f_1(x_1), f_2(x_2))
 \]
 stetig. Sind $f_1$ und $f_2$ offen, so ist auch $f_1 \times f_2$ offen.
\end{lem}
\begin{proof}
 Wir betrachten das kommutative Diagramm
 \begin{center}
  \begin{tikzpicture}[node distance = 7em, auto]
   \node (X_1 x X_2) {$X_1 \times X_2$};
   \node (X_1) [below left = 2em and 5em of X_1 x X_2] {$X_1$};
   \node (X_2) [below right = 2em and 5em of X_1 x X_2] {$X_2$};
   \node (T_1 x T_2) [below of = X_1 x X_2] {$T_1 \times T_2$};
   \node (T_1) [below of = X_1] {$T_1$};
   \node (T_2) [below of = X_2] {$T_2$};
   \draw[->] (X_1 x X_2) to node[swap]{$\pi_1$} (X_1);
   \draw[->] (X_1 x X_2) to node{$\pi_2$} (X_2);
   \draw[->] (X_1 x X_2) to node{$f_1 \times f_2$} (T_1 x T_2);
   \draw[->] (X_1) to node[swap]{$f_1$} (T_1);
   \draw[->] (X_2) to node{$f_2$} (T_2);
   \draw[->] (T_1 x T_2) to node{$\tau_1$} (T_1);
   \draw[->] (T_1 x T_2) to node[swap]{$\tau_2$} (T_2);
  \end{tikzpicture}
 \end{center}
 wobei $\pi_1, \pi_2, \tau_1$ und $\tau_2$ die entsprechenden kanonischen Projektionen bezeichnet. Da $\tau_1 \circ (f_1 \times f_2) = f_1 \circ \pi_1$ und $\tau_2 \circ (f_1 \times f_2) = f_2 \circ \pi_2$ stetig sind, ist es auch $f_1 \times f_2$ (siehe Aufgabe 3).
 
 Angenommen, $f_1$ und $f_2$ sind offen. Für offene Mengen $U \subseteq X_1$, $V \subseteq X_2$ ist dann auch $f_1(U) \subseteq T_1$ und $f_2(V) \subseteq T_2$ offen, also
 \[
  (f_1 \times f_2)(U \times V) = f_1(U) \times f_2(V) \subseteq T_1 \times T_2
 \]
 offen. Da die Mengen der Form $U \times V$ mit offenen Mengen $U \subseteq X_1$ und $V \subseteq X_2$ eine topologische Basis von $X_1 \times X_2$ bilden, zeigt dies die Offenheit von $f_1 \times f_2$.
\end{proof}


Für alle $j \in I$ bezeichne
\[
 \iota_j: X_j \to \coprod_{i \in I} X_i, x \mapsto (x,j)
\]
und
\[
 \iota'_j : X_j \times Y \to \coprod_{i \in I}(X_i \times Y), (x,y) \mapsto ((x,y),j)
\]
die entsprechenden kanonischen Inklusionen. Da $\iota_j$ für alle $j \in I$ stetig ist, ist nach Lemma \ref{lem: Funktionsprodukt} für alle $j \in I$ auch die Abbildung
\begin{align*}
 \iota_j \times \id_Y:
 X_j \times Y &\to \left(\coprod_{i \in I} X_i\right) \times Y \\
 (x,y) &\mapsto ((x,j),y).
\end{align*}
stetig. Deshalb gibt es nach der universellen Eigenschaft des Koproduktes (siehe Lemma \ref{lem: universell Koprodukt}) eine stetige Abbildung
\[
 f: \coprod_{i \in I} \left( X_i \times Y \right) \to \left( \coprod_{i \in I} X_i \right) \times Y,
\]
so dass das Diagramm
\begin{center}
 \begin{tikzpicture}[node distance = 6em, auto]
  \node (X_j x Y) {$X_j \times Y$};
  \node (coprod X_j x Y) [below left = 2 em and 1 em of X_j x Y] {$\coprod_{i \in I} (X_i \times Y)$};
  \node (coprod X_j x Y 2) [below right = 2 em and 1 em of X_j x Y] {$\left(\coprod_{i \in I} X_i\right) \times Y$};
  \draw[->] (X_j x Y) to node[swap]{$\iota'_j$} (coprod X_j x Y);
  \draw[->] (X_j x Y) to node{$\iota_j \times \id_Y$} (coprod X_j x Y 2);
  \draw[->] (coprod X_j x Y) to node{$f$} (coprod X_j x Y 2);
 \end{tikzpicture}
\end{center}
für alle $j \in I$ kommutiert. Dabei ist für alle $j \in I$ und $x \in X_j, y \in Y$
\[
 f( ((x,y),j) )
 = f( \iota_j'(x,y) )
 = (\iota_j \times \id_Y)(x,y)
 = ((x,j),y).
\]

$f$ ist offen: Seien $j \in I$ und $U \subseteq X_j \times Y$ offen beliebig aber fest. Da $\iota_j$ per Definition des Koproduktes offen ist, und die Identität $\id_Y$ offenbar ebenfalls offen ist, ist nach Lemma \ref{lem: Funktionsprodukt} auch $\iota_j \times \id_Y$ offen. Daher ist
\[
 f(U \times \{j\})
 = f(\iota_j'(U))
 = (\iota_j \times \id_Y)(U)
\]
offen.

Da die Mengen der Form $U \times \{j\} \subseteq \coprod_{i \in I} (X_i \times Y)$ mit $j \in I$ und $U \subseteq X_j$ offen eine topologische Basis von $\coprod_{i \in I}(X_i \times Y)$ bilden, zeigt dies die Offenheit von $f$.

Da $f$ offenbar auch bijektiv ist, ist $f$ ein Homöomorphismus.





\section{}

Für alle $j \in I$ bezeichnen wir die kanonische Projektion $\prod_{i \in I} X_i \to X_j$ mit $\pi_j$, und für alle $i \in I$ setzen wir $f_i := f \circ \pi_i$. Ist $f$ stetig, so ist $f_i$ als Verknüpfung stetiger Funktionen für alle $i \in I$ stetig.

Angenommen $f_i$ ist für alle $i \in I$ stetig. Für paarweise verschiedene Indizes $i_1, \ldots, i_n \in I$ und beliebige offene Mengen $U_1 \in X_{i_1}, \ldots, U_n \in X_{i_n}$ setzen wir
\[
 P_{i_1, \ldots, i_n}^{U_1, \ldots, U_n}
 = \prod_{i \in I} \begin{cases} U_k & \text{falls } i = i_k, \\ X_i & \text{sonst}, \end{cases}
 \subseteq \prod_{i \in I} X_i.
\]
Da die Mengen dieser Form eine topologische Basis von $\prod_{i \in I} X_i$ bilden, genügt es zum Nachweis der Stetigkeit von $f$ zu zeigen, dass
\[
 f^{-1}\left(P_{i_1, \ldots, i_n}^{U_1, \ldots, U_n}\right) \subseteq T
\]
offen ist für alle paarweise verschiedenen Indizes $i_1, \ldots, i_n \in I$ und beliebige offene Mengen $U_1 \in X_{i_1}, \ldots, U_n \in X_{i_n}$.

Seien also $i_1, \ldots, i_n$ und $U_1, \ldots, U_n$ wie zuvor beliebig aber fest. Wir bemerken, dass
\[
 P_{i_1, \ldots, i_n}^{U_1, \ldots, U_n} = \bigcap_{k=1}^n \pi_{i_k}^{-1}(U_k),
\]
und deshalb
\begin{align*}
 f^{-1}\left(P_{i_1, \ldots, i_n}^{U_1, \ldots, U_n}\right)
 &= f^{-1}\left(\bigcap_{k=1}^n \pi_{i_k}^{-1}(U_k)\right)
 = \bigcap_{k=1}^n (\pi_{i_k} \circ f)^{-1}(U_k)
 &= \bigcap_{k=1}^n f_{i_k}^{-1}(U_k).
\end{align*}
Da $U_k$ für alle $1 \leq k \leq n$ offen ist, und die $f_i$ alle stetig sind, ist $f_{i_k}^{-1}(U_k)$ für alle $1 \leq k \leq n$ offen, also $f^{-1}(P_{i_1, \ldots, i_n}^{U_1, \ldots, U_n})$ als endlicher Schnitt offener Mengen offen. Wegen der Beliebigkeit von $i_1, \ldots, i_n$ und $U_1, \ldots, U_n$ zeigt dies die Stetigkeit von $f$.





\section{}
\begin{lem}\label{lem: Verknüpfung erhält Homotopie}
 Seien $X', X, Y, Y'$ topologische Räume und $f,g,h,k$ stetige Abbildungen mit
 \begin{align*}
  f &: X \to Y, \\
  g &: X \to Y, \\
  h &: X' \to X \text{ und}\\
  k &: Y \to Y'.
 \end{align*}
 Ist $f \simeq g$, so ist auch $fh \simeq gh$ und $kf \simeq kg$.
\end{lem}
\begin{proof}
 Sei
 \[
  F : X \times [0,1] \to Y
 \]
 eine Homotopie mit $F(x,0) = f(x)$ und $F(x,1) = g(x)$ für alle $x \in X$. Da $h$ und $\id_{[0,1]}$ stetig sind, ist nach Lemma \ref{lem: Funktionsprodukt} auch
 \[
  h \times \id_{[0,1]} : X' \times [0,1] \to X \times [0,1].
 \]
 stetig. Daher ist auch die Verknüpfung
 \[
  F_h := F \circ (h \times \id_{[0,1]}) : X' \times [0,1] \to Y
 \]
 stetig. $F_h$ ist eine Homotopie und für alle $x' \in X'$ ist
 \begin{align*}
  F_h(x',0) &= F(h(x),0) = f(h(x')) = (fh)(x') \text{ und } \\
  F_h(x',1) &= F(h(x),1) = g(h(x')) = (gh)(x').
 \end{align*}
 Das zeigt, dass $fh \simeq gh$.
 
 Da $k$ und $F$ stetig sind, ist auch die Verknüpfung
 \[
  F_k := k \circ F : X \times [0,1] \to Y'. 
 \]
 stetig. $F_k$ ist eine Homotopie, und für alle $x \in X$ ist
 \begin{align*}
  F_k(x,0) &= k(F(x,0)) = k(f(x)) = (kf)(x) \text{ und} \\
  F_k(x,1) &= k(F(x,1)) = k(g(x)) = (kg)(x).
 \end{align*}
 Das zeigt, dass $kf \simeq kg$.
\end{proof}

Zunächst nehmen wir an, dass $\varphi : X \to Y$ ist eine Homotopieäquivalenz ist. Dann gibt es eine stetige Abbildung $\psi : Y \to X$ so dass
\[
 \psi \varphi \simeq \id_X \quad \text{und} \quad \varphi \psi \simeq \id_Y.
\]
Sei $K$ ein beliebiger topologischer Raum und
\[
 \varphi_* : [K,X] \to [K,Y], [f] \mapsto [\varphi f].
\]
$\varphi_*$ ist wohldefiniert, denn für $f,g : K \to X$ mit ist $f \simeq g$ ist nach Lemma \ref{lem: Verknüpfung erhält Homotopie} auch $\varphi f \simeq \varphi g$. Analog definieren wir
\[
 \psi_* : [K,Y] \to [K,X], [g] \mapsto [\psi g].
\]
Da $\psi \varphi \simeq \id_X$ ist nach Lemma \ref{lem: Verknüpfung erhält Homotopie}
\[
 \psi \varphi f \simeq \id_X f = f \text{ für alle } f : K \to X,
\]
und da $\varphi \psi \simeq \id_Y$ ist nach Lemma \ref{lem: Verknüpfung erhält Homotopie}
\[
 \varphi \psi g \simeq \id_Y g = g \text{ für alle } g : K \to Y.
\]
Also ist
\begin{align*}
 (\psi_* \varphi_*)([f]) &= [\psi \varphi f] = [f] \text{ für alle } [f] \in [K,X] \text{ und}\\
 (\varphi_* \psi_*)([g]) &= [\varphi \psi g] = [g] \text{ für alle } [g] \in [K,Y].
\end{align*}
Das zeigt, dass $\varphi_*$ bijektiv ist, und dass $\psi_* = \varphi_*^{-1}$.

Nun nehmen wir an, dass $\varphi : X \to Y$ stetig ist und für jeden topologischen Raum $K$ die induzierte Abbildung
\[
 \varphi^K_* : [K,X] \to [K,Y], [f] \mapsto [\varphi f].
\]
eine Bijektion ist. Wegen der Surjektivität von $\varphi_*^Y$ gibt existiert ein $\psi : Y \to X$ mit \mbox{$[\varphi \psi] = [\id_Y]$}, also $\varphi \psi \simeq \id_Y$. Da $\varphi \psi \simeq \id_Y$ ist nach Lemma \ref{lem: Verknüpfung erhält Homotopie}
\[
 \varphi_*^X([\psi \varphi]) = [\varphi \psi \varphi] = [\varphi] = \varphi_*^X([\id_X]),
\]
wegen der Injektivität von $\varphi_*^X$ also $[\psi \varphi] = [\id_X]$ und deshalb $\psi \varphi \simeq \id_X$. Das zeigt, dass $\varphi$ eine Homotopieäquivalenz ist.









\end{document} 
