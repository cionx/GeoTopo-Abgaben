\documentclass[a4paper,10pt]{article}
%\documentclass[a4paper,10pt]{scrartcl}

\usepackage{xltxtra}
\usepackage{../mystyle}

% die Nummer des Aufgabenblattes
\newcommand{\blattNummer}{6}

\setromanfont[Mapping=tex-text]{Linux Libertine O}
% \setsansfont[Mapping=tex-text]{DejaVu Sans}
% \setmonofont[Mapping=tex-text]{DejaVu Sans Mono}

\title{\sc Einführung in die Geometrie und Topologie \\ \Large Blatt 6}
\author{Jendrik Stelzner}
\date{\today}



\begin{document}
\maketitle





Im Folgenden schreiben wir $I := [0,1]$.





\section{(Freie versus punktierte Schlingen)}


\subsection{}
Es sei $x \in X$ beliebig aber fest. Es ist klar, dass die Inklusion
\[
 \left\{ f \,\middle|\, f : \left(S^1,1\right) \to (X,x) \text{ ist stetig} \right\}
 \hookrightarrow
 \left\{ g \,\middle|\, g : S^1 \to X \text{ ist stetig}\right\}
 \]
wohldefiniert ist. Zusammen mit der kanonischen Projektion
\[
 \left\{ g \,\middle|\, g : S^1 \to X \text{ ist stetig}\right\} \to \mc{S}(X), g \mapsto [g]
\]
ergibt sich damit eine wohldefinierte Abbildung
\[
 \left\{ f \,\middle|\, f : \left(S^1,1\right) \to (X,x) \text{ ist stetig} \right\} \to \mc{S}(X), f \mapsto [f].
\]
Um zu zeigen, dass diese über $\pi_1(X,x)$ faktorisiert, genügt es zu zeigen, dass für stetige Abbildungen $f,g : (S^1, 1) \to (X, x)$ mit $f \simeq g \rel 1$ auch $f \simeq g$. Dies ist aber klar.


\subsection{}

\begin{defi}
 Sei $X$ ein topologischer Raum und seien $v_1, \ldots, v_n : I \to X$ Wege mit $v_i(1) = v_{i+1}(0)$ für alle $i=1,\ldots,n-1$. Dann bezeichnen wir mit $v_1 * \cdots * v_n$ den Weg
 \[
  v_1 * \cdots * v_n : I \to X, t \mapsto
  \begin{cases}
   v_1(nt)       & \text{falls } 0 \leq t \leq \frac{1}{n}, \\
   v_2(nt-1)     & \text{falls } \frac{1}{n} \leq t \leq \frac{2}{n}, \\
   v_3(nt-2)     & \text{falls } \frac{2}{n} \leq t \leq \frac{3}{n}, \\
   \vdots        & \vdots \\
   v_n(nt-n+1) & \text{falls } \frac{n-1}{n} \leq t \leq 1.
  \end{cases}
 \]
\end{defi}

\begin{bem}
 Es ist bekannt und klar, dass die „Verknüpfung“ $*$ assoziativ bis auf Homotopie ist. Es ist auch klar, dass für eine Schlinge $C$ mit $C \simeq \const \rel 1$, bzw. $C \simeq \const \rel 0$, und einen Weg $v$ mit entsprechenden Anfangs-, bzw. Endpunkt
 \[
  C * v \simeq v \rel 1 \qquad \text{bzw.} \qquad v * C \simeq v \rel 0.
 \]
\end{bem}

\begin{lem}\label{lem: Kreisrotation}
 Sei $X$ ein topologischer Raum und seien $v_1, \ldots, v_n : I \to X$ Wege mit $v_i(1) = v_{i+1}(0)$ für alle $i=1,\ldots,n-1$ und $v_n(1) = v_1(0)$. Dann ist
 \[
  v_1 * v_2 * \cdots * v_n \simeq v_2 * \cdots * v_n * v_1.
 \]
\end{lem}
\begin{proof}
 Wir fassen die Schlingen $v_1 * v_2 * \cdots * v_n$ und $v_2 * \cdots * v_n * v_1$ in natürlicher Weise als Abbildungen
 \[
  f : (S^1, 1) \to (X, v_1(0)) \text{ und } g : (S^1,1) \to (X,v_2(0))
 \]
 auf. Dann ist $g(z) = f(e^{2\pi i/n}z)$ für alle $z \in S^1$ und eine entsprechende Homotopie durch
 \[
  F : S^1 \times I \to X, (z,t) \mapsto f\left( e^{t \cdot 2\pi i/n} z \right)
 \]
 gegeben.
\end{proof}

Sei nun erneut $x \in X$ beliebig aber fest. Es sei
\[
 \varphi : \pi_1(X,x) \to \mc{S}(X)
\]
die Vergiss-Abbildung.

Angenommen, $\varphi$ ist surjektiv. Dann gibt es für jeden Punkt $y \in X$ eine Schlinge $f : (S^1, 1) \to (X,x)$, so dass $f$ homotop zur konstanten Schlinge
\[
 g : S^1 \to X, z \mapsto y
\]
ist. Es gibt also eine Homotopie
\[
 F : S^1 \times I \to X
\]
mit $F(z,0) = f(z)$ und $F(z,1) = y$ für alle $z \in S^1$. Es ist daher
\[
 \gamma : I \to X, t \mapsto F(1,t)
\]
eine stetige Abbildung mit
\[
 \gamma(0) = F(1,0) = f(1) = x \text{ und } \gamma(1) = F(1,1) = y.
\]
Das zeigt, dass $X$ wegzusammenhängend ist.

Angenommen, $X$ ist wegzusammenhängend. Sei dann $f : I \to X$ eine Schlinge, also insbesondere $f(0) = f(1)$, beliebig aber fest. Da $X$ wegzusammenhängend ist, gibt es einen Weg $\gamma: I \to X$ von $x$ nach $f(0)$. Es sei $\gamma^{-1} : I \to X$ der umgekehrte Weg von $f(1)$ nach $x$, d.h. $\gamma^{-1}(t) = \gamma(1-t)$ für alle $t \in I$. Dann ist
\[
 g := \gamma * f * \gamma^{-1}
\]
eine Schlinge mit $g(0) = g(1) = x$, und nach Lemma \ref{lem: Kreisrotation} ist
\[
 g = \gamma * f * \gamma^{-1} \simeq f * \gamma^{-1} * \gamma \simeq f * (\gamma^{-1} * \gamma) \simeq f,
\]
da $\gamma^{-1} * \gamma$ relativ zu $f(0)$ zusammenziehbar ist. Also ist
\[
 \varphi([g]_{\pi_1(X,x)}) = [g] = [f].
\]
Wegen der Beliebigkeit von $f$ zeigt dies die Surjektivität von $\varphi$.


\subsection{}
Auch hier sei $x \in X$ beliebig aber fest. Es seien $f, g : (I, \del I) \to (X,x)$, so dass $[f]_{\pi_1(X,x)}$ und $[g]_{\pi_1(X,x)}$ konjugiert zueinander sind, d.h. dass es eine Schlinge $h : (I, \del I) \to (X,x)$ gibt mit
\[
 [f]_{\pi_1(X,x)} = [h]_{\pi_1(X,x)} [g]_{\pi_1(X,x)} [h]_{\pi_1(X,x)}^{-1} = \left[h * g * h^{-1}\right]_{\pi_1(X,x)}.
\]
Es ist dann nach Lemma \ref{lem: Kreisrotation}
\begin{align*}
 [f]
 &= \varphi([f]_{\pi_1(X,x)})
 = \varphi\left(\left[g * h * g^{-1}\right]_{\pi_1(X,x)} \right)
 = \left[h * g * h^{-1}\right] \\
 &= \left[g * h^{-1} * h \right]
 = [g],
\end{align*}
da $h^{-1} * h$ relativ zu $g(1)$ zusammenziehbar ist.

Andererseits seien $f,g : (I, \del I) \to (X,x)$ Schlingen, so dass
\[
 [f]_{\pi_1(X,x)} = [g]_{\pi_1(X,x)},
\]
also $f \simeq g$. Dann gibt es eine Homotopie $F : I \times I \to X$ mit $F(t,0) = f(t)$ und $F(t,1) = g(t)$ für alle $t \in I$, wobei zusätzlich
\[
 F(0,s) = F(1,s) \text{ für alle } s \in I.
\]
Für alle $s \in I$ definieren wir
\[
 \gamma_s : I \to X \text{ mit } \gamma_s(t) := F(0,ts).
\]
Für alle $s \in I$ ist
\[
 \gamma_s(0) = F(0,0) = f(0) = x \text{ und }
 \gamma_s(1) = F(0,s),
\]
also $\gamma_s$ ein Weg von $x$ zu $F(0,s)$. Auch ist
\[
 \gamma_1(1) = F(0,1) = g(1) = x,
\]
also $\gamma_1$ eine Schlinge mit $\gamma_1(0) = \gamma_1(1) = x$. Für alle $s \in I$ definieren wir auch
\[
 \gamma_s^{-1} : I \to X, t \mapsto \gamma_s(1-t).
\]
Wir betrachten die Homotopie $G : I \times I \to X$ mit
\[
 G(t,s) :=
 \begin{cases}
  \gamma_s\left( \frac{3}{s} t \right)             & \text{für } 0 \leq t < \frac{s}{3}, \\
  F\left( \frac{3t-s}{3-2s}, s\right)              & \text{für } \frac{s}{3} \leq t \leq 1-\frac{s}{3}, \\
  \gamma_s^{-1}\left( \frac{3}{s}(t-1) + 1 \right) & \text{für } 1-\frac{s}{3} < t \leq 1.
 \end{cases}
\]
(Die Wohldefiniertheit und Stetigkeit von $G$ ist klar.) Für alle $t \in I$ ist
\[
 G(t,0) = F(t,0) = f(t) \text{ und } G(t,1) = \left( \gamma_1 * g * \gamma_1^{-1} \right)(t).
\]
Außerdem ist $G(0,s) = G(1,s) = x$ für alle $s \in I$. Das zeigt, dass
\[
 f \simeq \gamma_1 * g * \gamma_1^{-1} \rel \del I,
\]
also dass
\[
 [f]_{\pi_1(X,x)}
 = \left[ \gamma_1 * g * \gamma_1^{-1} \right]_{\pi_1(X,x)}
 = \left[ \gamma_1 \right]_{\pi_1(X,x)} \left[ g \right]_{\pi_1(X,x)} \left[ \gamma_1 \right]^{-1}_{\pi_1(X,x)}.
\]





\section{}


\subsection{}
Da $A$ und $B$ wegzusammenhängend sind, und $A \cap B \neq \emptyset$, ist $X = A \cup B$ wegzusammenhängend. Insbesondere ist daher
\[
 \pi_1(X,x) \cong \pi_1(X,x') \text{ für alle } x, x' \in X.
\]
Um zu zeigen, dass $\pi_1(X,x)$ für alle $x \in X$ trivial ist, genügt es daher zu zeigen, dass $\pi_1(X,x)$ für irgendein $x \in X$ trivial ist. Sei hierfür $x \in A \cap B \neq \emptyset$.

Es sei $f : (I, \del I) \to (X,x)$ eine Schlinge. Da $A$ und $B$ offen sind, und $X = A \cup B$, ist $\{f^{-1}(A), f^{-1}(B)\}$ wegen der Stetigkeit von $f$ eine offene Überdeckung von $I$. Da $I$ ein kompakter metrischer Raum ist, gibt es nach dem Lemma von Lebesgue ein $\delta > 0$, so dass jede Teilmenge $M \subseteq I$ mit $\diam M < \delta$ komplett in $f^{-1}(A)$ oder $f^{-1}(B)$ enthalten ist. Sei $m \in \N, m \geq 1$ groß genug, dass die Teilintervalle
\[
 I_k := \left[ \frac{k-1}{m}, \frac{k}{m} \right] \subseteq I \text{ für } k = 1, \ldots, m
\]
je komplett in $f^{-1}(A)$ oder $f^{-1}(B)$ enthalten ist. Dann ist $f(I_k) \subseteq A$ oder $f(I_k) \subseteq B$ für alle $k = 1, \ldots, m$.

Wir definieren
\[
 x_k := f\left( \frac{k}{m} \right) \text{ für alle } k = 0, \ldots, m.
\]
Inbesondere ist
\[
 x_0 = x_m = x.
\]
Da $A$, $B$ und $A \cap B$ wegzusammenhängend sind, gibt es für alle $k = 1, \ldots, m-1$ einen Weg
\[
 \lambda_k : I \to X
\]
von $x$ nach $x_k$, der je komplett in $A$ verläuft, wenn $x_k \in A$, bzw. komplett in $B$ verläuft, wenn $x_k \in B$, bzw. komplett in $A \cap B$ verläuft, wenn $x_k \in A \cap B$. Für alle $k = 1, \ldots, m$ definieren wir außerdem den Weg $f_k : I \to X$ mit
\[
 f_k(t) := f\left( \frac{k-1}{m} + \frac{t}{m} \right) \text{ für alle } t \in I
\]
von $x_{k-1}$ nach $x_k$. Man bemerke, dass  $\Img f_k = f(I_k)$ für alle $k = 1, \ldots, m$.
Es ist nun offenbar
\begin{align*}
 &\, f \\
 =&\, f_1 * f_2 * \cdots * f_m \\
 \simeq&\, f_1 * \left( \lambda_1^{-1} \lambda_1 \right) * f_2 * \left( \lambda_2^{-1} \lambda_2 \right) * f_3 * \cdots * f_{m-1} * \left( \lambda_{m-1}^{-1} \lambda_{m-1} \right) * f_m \rel x\\
 \simeq&\, \left( f_1 * \lambda_1^{-1} \right) * \left( \lambda_1 * f_2 * \lambda_2^{-1} \right) * \cdots * \left( \lambda_{m-2} * f_{m-1} * \lambda_{m-1}^{-1} \right) \\
  &\, * \left( \lambda_{m-1} * f_m \right) \rel x.
\end{align*}
Die Wege $f_1 * \lambda_1^{-1}, \lambda_1 * f_2 * \lambda_2^{-1}, \ldots, \lambda_{m-2} * f_{m-1} * \lambda_{m-1}^{-1}$ und $\lambda_{m-1} * f_m$ jeweils Schleifen, die von $x$ ausgehen, und sich jeweils komplett in $A$ oder $B$ befinden. Daher sind sie alle zusammenziehbar relativ zu $x$, denn $A$ und $B$ sind einfach zusammenhängend. Daher ist auch $f$ relativ zu $x$ zusammenziehbar.

Aus der Beliebigkeit von $f$ folgt, dass $\pi_1(X,x)$ trivial ist.


\subsection{}
Es sei $n > 1$. Wir schreiben $S^n = A \cup B$ mit
\begin{align*}
 A &= \left\{ (x_1, \ldots, x_{n+1}) \in S^n \,\middle|\, x_{n+1} \neq 1 \right\} \text{ und} \\
 B &= \left\{ (x_1, \ldots, x_{n+1}) \in S^n \,\middle|\, x_{n+1} \neq -1 \right\}.
\end{align*}
$A$ und $B$ sind offenbar offen in $S^n$, und durch die entsprechenden stereografischen Projektionen wissen wir, dass $A \cong \R^n$ und $B \cong \R^n$. Da $\R^n$ offenbar einfach zusammenhängend ist, sind es daher auch $A$ und $B$ (wegen der Funktorialität von $\pi_1$). Offenbar ist $A \cap B \neq \emptyset$ wegzusammenhängend (hier benutzen wir, dass $n > 1$). Nach Aufgabenteil (a) ist deshalb auch $S^n$ einfach zusammenhängend.





\section{}
Es sei $(x,x') \in X \times X'$ beliebig aber fest. Da $p : E \to X$ eine Überlagerung ist, gibt es eine Umgebung $U \subseteq X$ von $x$, einen diskreten Raum $D$ und einen Homöomorphismus $\phi : p^{-1}(U) \to U \times D$, so dass das Diagramm in Abbildung \ref{fig: Überlagerung für U} kommutiert. 
\begin{figure}[h]\centering
 \begin{tikzpicture}[auto]
  \node (p^-1 U) {$p^{-1}(U)$};
  \node (U x D) [right of = p^-1 U, node distance = 9em] {$U \times D$};
  \node (U1) [below of = p^-1 U, node distance = 5em] {$U$};
  \node (U2) [below of = U x D, node distance = 5em] {$U$};
  \draw[->] (p^-1 U) to node{$\phi$} (U x D);
  \draw[->] (U1) to node[swap]{$\id_U$} (U2);
  \draw[->] (p^-1 U) to node[swap]{$p$} (U1);
  \draw[->] (U x D) to node{$\pi_1$} (U2);
 \end{tikzpicture}
 \caption{$p : E \to X$ ist eine Überlagerung.}
 \label{fig: Überlagerung für U}
\end{figure}

Analog gibt es, da $p' : E' \to X'$ eine Überlagerung ist, eine Umgebung $U' \subseteq X'$ von $x'$, einen diskreten Raum $D'$ und einen Homöomorphismus $\phi' : p'^{-1}(U) \to U' \times D'$, so dass das Diagramm in Abbildung \ref{fig: Überlagerung für U'} kommutiert.
\begin{figure}[h]\centering
 \begin{tikzpicture}[auto]
  \node (p^-1 U) {$p'^{-1}(U)$};
  \node (U x D) [right of = p^-1 U, node distance = 9em] {$U' \times D'$};
  \node (U1) [below of = p^-1 U, node distance = 5em] {$U'$};
  \node (U2) [below of = U x D, node distance = 5em] {$U'$};
  \draw[->] (p^-1 U) to node{$\phi'$} (U x D);
  \draw[->] (U1) to node[swap]{$\id_{U'}$} (U2);
  \draw[->] (p^-1 U) to node[swap]{$p'$} (U1);
  \draw[->] (U x D) to node{$\pi'_1$} (U2);
 \end{tikzpicture}
 \caption{$p' : E' \to X'$ ist eine Überlagerung.}
 \label{fig: Überlagerung für U'}
\end{figure}

Es ist nun $U \times U' \subseteq X \times X'$ eine Umgebung von $(x,x')$ in $X \times X'$ mit
\[
 \left( p \times p' \right)^{-1} (U \times U') = p^{-1}(U) \times p'^{-1}(U').
\]
Die Homöomorphismen $\phi$ und $\phi'$ liefern uns einen Homöomorphismus
\[
 \tilde{\phi} :
 p^{-1}(U) \times p'^{-1}(U')
 \cong (U \times D) \times (U' \times D')
 \cong (U \times U') \times (D \times D').
\]
Zusammen liefert dies das kommutative Diagramm in Abbildung \ref{fig: Überlagerung für U x U'}. Dabei ist klar, dass $D \times D'$ ebenfalls diskret ist.
\begin{figure}[ht]\centering
 \begin{tikzpicture}[auto]
  \node (p^-1 U) {$(p \times p')^{-1}(U \times U')$};
  \node (U x D) [right of = p^-1 U, node distance = 12em] {$(U \times U') \times (D \times D')$};
  \node (U1) [below of = p^-1 U, node distance = 5em] {$U \times U'$};
  \node (U2) [below of = U x D, node distance = 5em] {$U \times U'$};
  \draw[->] (p^-1 U) to node{$\tilde{\phi}$} (U x D);
  \draw[->] (U1) to node[swap]{$\id_{U \times U'}$} (U2);
  \draw[->] (p^-1 U) to node[swap]{$p \times p'$} (U1);
  \draw[->] (U x D) to node{$\tilde{\pi}_1$} (U2);
 \end{tikzpicture}
 \caption{$p' : E' \to X'$ ist eine Überlagerung.}
 \label{fig: Überlagerung für U x U'}
\end{figure}

Wegen der Beliebigkeit von $(x,x') \in X \times X'$ zeigt dies, dass
\[
 p \times p' : E \times E' \to X \times X'
\]
eine Überlagerung ist. Es ist auch klar, dass sich die Blätterzahl von $(x,x') \in X \times X'$ als Produkt der Blätterzahl von $x$ und der Blätterzahl von $x'$ ergibt.


\subsection{}
Aus der Definition von $Y \times_X E$ erhalten wir das kommutative Diagramm in Abbildung \ref{fig: Faserprodukt}.
\begin{figure}[ht]\centering
 \begin{tikzpicture}[auto]
  \node (Y x E) {$Y \times_X E$};
  \node (Y) [right of = Y x E, node distance = 8em] {$Y$};
  \node (E) [below of = Y x E, node distance = 5em] {$E$};
  \node (X) [right of = E, node distance = 8em] {$X$};
  \draw[->] (Y x E) to node {$\pi_1$} (Y);
  \draw[->] (E) to node[swap] {$p$} (X);
  \draw[->] (Y x E) to node[swap] {$\pi_2$} (E);
  \draw[->] (Y) to node {$f$} (X);
 \end{tikzpicture}
 \caption{Das Faserprodukt $Y \times_X E$.}
 \label{fig: Faserprodukt}
\end{figure}

Se $y \in Y$ beliebig aber fest. Da $p : E \to X$ eine Überdeckung ist, gibt es eine Umgebung $U \subseteq X$ von $f(y) \in X$, einen diskreten Raum $D$ und einen Homöomorphismus $\phi : p^{-1}(U) \to U \times D$, so dass das Diagramm in Abbildung \ref{fig: unterer Teil} kommutiert.
\begin{figure}[ht]\centering
 \begin{tikzpicture}[auto]
  \node (p^-1 U) {$p^{-1}(U)$};
  \node (U x D) [right of = p^-1 U, node distance = 9em] {$U \times D$};
  \node (U1) [below of = p^-1 U, node distance = 5em] {$U$};
  \node (U2) [below of = U x D, node distance = 5em] {$U$};
  \draw[->] (p^-1 U) to node{$\phi$} (U x D);
  \draw[->] (U1) to node[swap]{$\id_U$} (U2);
  \draw[->] (p^-1 U) to node[swap]{$p$} (U1);
  \draw[->] (U x D) to node{$\pr_1$} (U2);
 \end{tikzpicture}
 \caption{$p : E \to X$ ist eine Überlagerung.}
 \label{fig: unterer Teil}
\end{figure}

Wir erweitern das Diagramm in Abbildung \ref{fig: unterer Teil} zu dem Diagramm in Abbildung \ref{fig: gesammtes Diagramm}. Dabei wollen wir den Homöomorphismus $\phi$ zu einem Homöomorphismus $\Phi$ liften.
\begin{figure}[ht]\centering
 \begin{tikzpicture}[auto]
  \node (p^-1 U) {$p^{-1}(U)$};
  \node (U x D) [right of = p^-1 U, node distance = 10em] {$U \times D$};
  \node (U1) [below of = p^-1 U, node distance = 5em] {$U$};
  \node (U2) [below of = U x D, node distance = 5em] {$U$};
  \node (pi^-1 p^-1 U) [above of = p^-1 U, node distance = 5em] {$\pi_2^{-1}(p^{-1}(U))$};
  \node (f^-1 U x D) [above of = U x D, node distance = 5em] {$f^{-1}(U) \times D$};
  \draw[->] (p^-1 U) to node{$\phi$} (U x D);
  \draw[->] (U1) to node[swap]{$\id_U$} (U2);
  \draw[->] (p^-1 U) to node[swap]{$p$} (U1);
  \draw[->] (U x D) to node{$\pr_1$} (U2);
  \draw[->] (pi^-1 p^-1 U) to node[swap]{$\pi_2$} (p^-1 U);
  \draw[->] (f^-1 U x D) to node{$f \times \id_D$} (U x D);
  \draw[->,dashed] (pi^-1 p^-1 U) to node{$\Phi$} (f^-1 U x D);
 \end{tikzpicture}
 \caption{$p : E \to X$ ist eine Überlagerung.}
 \label{fig: gesammtes Diagramm}
\end{figure}

Hierfür definieren wir
\[
 \Phi : \pi_2^{-1}(p^{-1}(U)) \to f^{-1}(U) \times D, (y,e) \mapsto \left( y, \pr_2 \phi(e) \right),
\]
wobei $\pr_2 : U \times D \to D$ die kanonische Projektion bezeichnet. $\Phi$ ist wohldefiniert, denn für $(y,e) \in \pi_2^{-1}(p^{-1}(U))$ ist
\[
 f(y) = p(e) = p \pi_2(y,e) \in U.
\]
Außerdem definieren wir
\[
 \Psi : f^{-1}(U) \times D \to \pi_2^{-1}(p^{-1}(U)), (y,d) \mapsto \left( y, \phi^{-1}(f(y),d) \right).
\]
$\Psi$ ist wohldefiniert, denn für $(f,d) \in f^{-1}(U) \times D$ ist
\[
 p\left(\pi_2\left( y, \phi^{-1}(f(y),d) \right) \right)
 = p\left( \phi^{-1}(f(y),d) \right)
 = \pr_1(f(y),d)
 = f(y)
 \in U.
\]

Es ist klar, dass $\Phi$ und $\Psi$ stetig sind. Wir bemerken, dass $\Phi$ und $\Psi$ invers zueinander sind. Für $(y,e) \in \pi_2^{-1}(p^{-1}(U))$ ist
\[
 \Psi \Phi (y,e)
 = \Psi \left( y, \pr_2 \phi(e) \right)
 = \left( y, \phi^{-1}(f(y), \pr_2 \phi(e)) \right),
\]
wobei
\[
 e
 = \phi^{-1} \phi e
 = \phi^{-1}\left( p(e), \pr_2 \phi(e) \right)
 = \phi^{-1}\left( f(y), \pr_2 \phi(e) \right).
\]
Für $(y,d) \in f^{-1}(U) \times D$ ist
\[
 \Phi \Psi (y,d)
 = \Phi\left( y, \phi^{-1}(f(y),d) \right)
 = \left( y, \pr_2 \phi \phi^{-1}(f(y),d) \right)
 = (y,d).
\]
Also ist $\Phi$ ein Homöomorphismus. Das Diagramm in Abbildung \ref{fig: gesammtes Diagramm} kommutiert auch, denn für $(y,e) \in \pi_2^{-1}(p^{-1}(U))$ ist
\begin{align*}
 \phi \pi_2 (y,e)
 &= \phi(e)
 = (p(e), \pr_2 \phi(e))
 = (f(y), \pr_2 \phi(e)) \\
 &= (f \times \id_D) (y, \pr_2 \phi(e))
 = (f \times \id_D) \Phi (y,e).
\end{align*}

Wegen der Kommutativität des Diagramms in Abbildung \ref{fig: Faserprodukt} ist $p \pi_2 = f \pi_1$, also
\[
 \pi_2^{-1}(p^{-1}(U)) = \pi_1^{-1}(f^{-1}(U)).
\]
Da $f$ stetig ist und $U$ eine Umgebung von $f(y)$ ist, ist $f^{-1}(U)$ eine Umgebung von $y$. Damit erhalten wir das Diagramm in \ref{fig: Überlagerung für f^-1 U}.
\begin{figure}[ht]\centering
 \begin{tikzpicture}[auto]
  \node (pi^-1 f^-1 U) {$\pi_1^{-1}(f^{-1}(U))$};
  \node (f^-1 U x D) [right of = pi^-1 f^-1 U, node distance = 12em] {$f^{-1}(U) \times D$};
  \node (f^-1 U 1) [below of = pi^-1 f^-1 U, node distance = 5em] {$f^{-1}(U)$};
  \node (f^-1 U 2) [below of = f^-1 U x D, node distance = 5em] {$f^{-1}(U)$};
  \draw[->] (pi^-1 f^-1 U) to node{$\Phi$} (f^-1 U x D);
  \draw[->] (f^-1 U 1) to node[swap]{$\id_{f^-1(U)}$} (f^-1 U 2);
  \draw[->] (pi^-1 f^-1 U) to node[swap]{$\pi_1$} (f^-1 U 1);
  \draw[->] (f^-1 U x D) to node{$\tau_1$} (f^-1 U 2);
 \end{tikzpicture}
 \caption{$\pi_1 : Y \times_X E \to Y$ ist eine Überlagerung.}
 \label{fig: Überlagerung für f^-1 U}
\end{figure}
Dieses kommutiert, denn für alle $(y,e) \in \pi_1^{-1}(f^-1(U))$ ist
\[
 \pi_1((y,e)) = y = \tau_1 (y, \pr_2 \phi(e)) = \tau_1 \Phi(y,e).
\]
Wegen der Beliebigkeit von $y \in Y$ zeigt dies, dass $\pi_1 : Y \times_X E \to Y$ eine Überlagerung ist. Aus der obigen Konstruktion geht direkt hervor, dass die Blätterzahl von $y \in Y$ bezüglich dieser Überlagerung der Blätterzahl von $f(y) \in X$ bezüglich der Überdeckung $p : E \to X$ entspricht.





\section{}
Wir bemerken zunächst, dass für jeden topologischen Raum $E$ und jede Gruppe $G$, die stetig (von rechts) auf $X$ wirkt, für jedes $g \in G$ die Abbildung
\[
 \pi_g : E \to E, e \mapsto eg
\]
ein Homöomorphismus ist, denn $\pi_g$ ist bekanntermaßen bijekitv, und
\[
 \pi_g^{-1} = \pi_{g^{-1}}
\]
ist ebenfalls stetig.


\subsection{}
Ist $G = 1$, so ist nichts zu zeigen. Ansonsten seien $e_0 \in E$ und $g \in G-\{1\}$ beliebig aber fest.

Da $\pi_g : E \to E$ stetig ist, ist auch die Abbildung
\[
 f : E \to E \times E, (e, \mapsto eg) = \left( \id_E(e), \pi_g(e) \right)
\]
stetig. Da $g \neq 1$ und $G$ frei auf $E$ wirkt, ist $\Img f \cap \Delta = \emptyset$, wobei $\Delta \subseteq E \times E$ die Diagonale bezeichnet. Da $E$ Hausdorff ist, ist $\Delta$ abgeschlossen in $E \times E$, also $E \times E - \Delta$ offen in $E \times E$. Da $(e_0, e_0 g) \in \Img f \subseteq E \times E - \Delta$ gibt es deshalb $U, V \subseteq E$ offen mit $(e_0, e_0g) \in U \times V \subseteq E \times E - \Delta$. Inbesondere ist $U \times V \cap \Delta = \emptyset$, also $U \cap V = \emptyset$.

Da $\pi_{g^-1}$ ein Homöomorphismus ist, ist $\pi_{g^{-1}}(V) = Vg^{-1}$ offen, und da $e_0 g \in V$ ist $e_0 \in Vg^{-1}$. Wir setzen
\[
 W = U \cap Vg^{-1}.
\]
$W$ ist eine offene Umgebung von $e_0$ mit
\begin{align*}
 W \cap Wg
 &= \left( U \cap Vg^{-1} \right) \cap \left( U \cap Vg^{-1} \right)g \\
 &= U \cap Vg^{-1} \cap Ug \cap V
 \subseteq U \cap V
 = \emptyset,
\end{align*}
also $W \cap Wg = \emptyset$.

Wir finden also für jedes $g \in G-\{1\}$ eine offene Umgebung $U_g$ von $e_0$ mit
\[
 U_g \cap U_g g = \emptyset.
\]
Wir setzen
\[
 U := \bigcap \{U_g \mid g \in G-\{1\}\}.
\]
Da $G$ endlich ist, ist auch $U$ ein offene Umgebung von $e_0$, und für alle $g \in G-\{1\}$ ist
\[
 U \cap Ug
 \subseteq U_g \cap U_g g
 = \emptyset,
\]
also $U \cap Ug = \emptyset$. Wegen der Beliebigkeit von $e_0 \in E$ zeigt dies, dass $G$ eigentlich diskontinuierlich auf $E$ wirkt.

Fun fact: Die Aufgabe lässt sich auch auf intuitive Weise mithilfe von Netzen lösen. Wir setzen
\[
 D := \{U \subseteq E \mid U \text{ ist eine Umgebung von } E\}
\]
und ordnen $D$ via
\[
 U \geq V \Leftrightarrow U \subseteq V,
\]
und erhalten so eine geordnete Menge. Ist $U \cap Ug = \emptyset$ für jedes $U \in D$, so gibt es für jedes $U \in D$ Elemente $x_D, y_D \in D$ mit $y_U = x_U g = \pi_g(X_U)$. Für die beiden Netze $(x_U)_{U \in D}$o und $(y_U)_{U \in D}$ ist klar, dass $x_u \to e_0$ und $y_0 \to e_0$. Da $\pi_g$ stetig ist, ist deshalb auch
\[
 y_U = x_U g = \pi_g(x_U) \to \pi_g(e_0) = e_0 g.
\]
Da $E$ Hausdorff ist, sind Grenzwerte von Netzen eindeutig. Daher ist $e_0 = e_0 g$. Dies steht wegen $g \neq 1$ im Widerspruch dazu, dass $G$ frei auf $E$ wirkt. Also muss es ein $U \in D$ geben mit $U \cap Ug = \emptyset$.


\subsection{}
Es sei $p : E \to E/G$ die kanonische Projektion. Seien $y_0 \in E/G$ und $x_0 \in E$ mit $p(x_0) = y_0$ beliebig aber fest. Da $G$ eigentlich diskontinuierlich auf $E$ wirkt, gibt es eine offene Umgebung $U$ von $x_0$, so dass $U \cap Ug = \emptyset$ für alle $g \in G-\{1\}$.

Wir bemerken, dass für $g_1, g_2 \in G$ mit $g_1 \neq g_2$ auch $U g_1 \cap U g_2 = \emptyset$, da
\[
 \pi_{g_1^{-1}}(U g_1 \cap U g_2) = U \cap U g_2 g_1^{-1} = \emptyset. 
\]

Für $V := p(U)$ ist daher
\[
 p^{-1}(V) = p^{-1}(p(U)) = U^{\text{sat}} = UG = \bigdotcup_{g \in G} Ug.
\]

Dabei ist klar, dass
\[
 \bigdotcup_{g \in G} Ug
 \cong \coprod_{g \in G} Ug
 \cong \coprod_{g \in G} U
 \cong U \times G,
\]
wobei wir $G$ als diskreten Raum auffassen. (Man beachte, dass $\pi_g$ einen Homöomorphismus $U \cong Ug$ induziert. Der Homöomorphismus $U \times G \cong \bigdotcup_{g \in G} Ug$ durch
\[
 \xi : U \times G \to \bigdotcup_{g \in G} Ug, (u,g) \mapsto ug
\]
gegeben.

Wir bemerken, dass sich $p$ zu einem Hömomorphismus $\tilde{p} : U \cong V$ beschränkt: Die Surjektivität und Stetigkeit sind klar. Die Injektivität von $\tilde{p}$ ergibt sich daraus, dass für $x, x' \in U$ mit $\tilde{p}(x) = \tilde{p}(x')$ ein $g \in G$ existiert mit $x' = xg$, wegen $U \cap Ug' = \emptyset$ für $g' \in G-\{1\}$ also $g = 1$ und damit $x' = x$ ist. $\tilde{p}$ ist offen, denn für eine offene Teilmenge $W \subseteq U$ ist $W$ auch offen in $E$, da $U$ offen ist, also auch
\[
 \pi^{-1}(p(W)) = WG = \bigdotcup_{g \in G} Wg
\]
offen in $E$, und mati $p(W) = \tilde{p}(W)$ offen in $E/G$, also auch offen in $p(V)$.

Damit ist auch
\[
 \bigdotcup_{g \in G} Ug \cong V \times G
\]
durch den Homöomorphismus
\[
 \zeta : V \times G \to \bigdotcup_{g \in G} Ug, (v,g) \mapsto \tilde{p}^{-1}(v) g.
\]
Damit ergibt sich insgesamt das kommutierende Diagramm in Abbildung \ref{fig: Überlagerung für V}.
\begin{figure}[ht]\centering
 \begin{tikzpicture}[auto]
  \node (p^-1 V) {$p^{-1}(U)$};
  \node (V x G) [right of = p^-1 V, node distance = 9em] {$V \times D$};
  \node (V1) [below of = p^-1 V, node distance = 5em] {$V$};
  \node (V2) [below of = V x G, node distance = 5em] {$V$};
  \draw[->] (p^-1 V) to node{$\zeta^{-1}$} (V x G);
  \draw[->] (V1) to node[swap]{$\id_V$} (V2);
  \draw[->] (p^-1 V) to node[swap]{$p$} (V1);
  \draw[->] (V x G) to node{$\pi_1$} (V2);
 \end{tikzpicture}
 \caption{$p : E \to E/G$ ist eine Überlagerung.}
 \label{fig: Überlagerung für V}
\end{figure}
(Das Diagramm kommutiert, da für alle $(v,g) \in V \times G$
\[
 \pi_1(v,g) = v
\]
und
\[
 p \zeta(v,g) = p(\tilde{p}^{-1}(v) g) = p(\tilde{p}^{-1}(v)) = \tilde{p}(\tilde{p}^{-1}(v)) = v,
\]
also $\pi_1 = p \zeta$ und somit $\pi_1 \zeta^{-1} = p$.)

Aus der Beliebigkeit von $y_0 \in E/G$ folgt, dass $p : E \to E/G$ eine Überlagerung ist. Die Blätterzahl ist offenbar $|G|$.


\subsection{}
Die Gruppe $G = \Z/2 = \{1,-1\}$ wirkt stetig von rechts auf $S^n$ via
\[
 x.1 = x \text{ und } x.(-1) = -x \text{ für alle } x \in S^n.
\]
Die induzierte Äquivalenzrelation $\sim$ auf $S^n$ identifiziert jeden Punkt mit seinem Antipodenpunkt. Da $G$ endlich ist, die offenbar stetig und frei auf dem Hausdorff-Raum $S^n$ wirkt, ist die Gruppenwirkung nach Aufgabenteil (a) eigentlich diskontinuierlich. Nach Aufgabenteil (b) ist daher die Quotientenabbildung $p : S^n \to S^n/{\sim}$ eine zweiblättrige Überlagerung. Da diese der Quotientenabbildung $S^n \to \R P^n$ entspricht, zeigt dies die Aussage.





























\end{document} 
