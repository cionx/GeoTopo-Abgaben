\documentclass[a4paper,10pt]{article}
%\documentclass[a4paper,10pt]{scrartcl}

\usepackage{xltxtra}
\usepackage{../mystyle}

% die Nummer des Aufgabenblattes
\newcommand{\blattNummer}{5}

\setromanfont[Mapping=tex-text]{Linux Libertine O}
% \setsansfont[Mapping=tex-text]{DejaVu Sans}
% \setmonofont[Mapping=tex-text]{DejaVu Sans Mono}

\title{\sc Einführung in die Geometrie und Topologie \\ \Large Blatt 5}
\author{Jendrik Stelzner}
\date{\today}



\begin{document}
\maketitle





\section{}
Wir nehmen zunächst an, dass $f : S^1 \to X$ homotop zu einer konstanten Schlinge ist. Dann gibt es eine Homotopie $F : S^1 \times [0,1] \to X$ mit
\begin{align*}
 F(s,0) &= \const \text{ und} \\
 F(s,1) &= f(s)
\end{align*}
für alle $s \in S$. Es sei
\begin{gather*}
 A := S^1 \times \{0\} \subseteq S^1 \times [0,1]
\shortintertext{und}
 \pi : S^1 \times [0,1] \to \left( S^1 \times [0,1] \right) / A
\end{gather*}
die kanonische Projektion.

Es ist klar, dass $F$ als Abbildung
\[
 \tilde{F} : \left( S^1 \times [0,1] \right) / A \to X
\]
faktorisiert. Diese ist nach der universellen Eigenschaft des Quotienten stetig.

Wir bemerken weiter, dass $\left(S^1 \times [0,1] \right) / A \cong D^2$: Die Abbildung
\[
 \phi : S^1 \times [0,1] \to D^2, (s,t) \mapsto ts
\]
ist stetig und surjektiv, und faktorisiert als Bijektion
\[
 \psi : \left( S^1 \times [0,1] \right) / A \to D^2,
\]
die nach der universellen Eigenschaft des Quotienten stetig ist. Da $S^1 \times [0,1]$ als Produkt kompakter Räume kompakt ist, ist auch $\left( S^1 \times [0,1] \right) / A$ quasikompakt. Da $D^2$ Hausdorff ist, ist damit $\psi$ bereits ein Homöomorphismus.

Durch den Isomorphismus $\psi$ faktorisiert $\tilde{F}$ als stetige Abbildung $\bar{F} : D^2 \to X$.

Insgesamt ergibt sich damit ein Diagramm wie in Abbildung \ref{fig: wow such commute}.
\begin{figure}\centering
 \begin{tikzpicture}[auto]
  \node (X) {$X$};
  \node (Product) [above left = 2em and 5em of X] {$S^1 \times [0,1]$};
  \node (Quotient) [above right = 2em and 5em of X] {$\left( S^1 \times [0,1] \right) / A$};
  \node (D2) [below of = X, node distance = 6em] {$D^2$};
  \draw[->] (Product) to node {$\pi$} (Quotient);
  \draw[->] (Product) to node {$F$} (X);
  \draw[->] (Quotient) to node[swap] {$\tilde{F}$} (X);
  \draw[->] (Product) to node[swap] {$\phi$} (D2);
  \draw[->] (Quotient) to node {$\psi$} (D2);
  \draw[->] (D2) to node {$\bar{F}$} (X);
 \end{tikzpicture}
 \caption{Mir fällt kein passender Titel ein.}
 \label{fig: wow such commute}
\end{figure}
Dieses Diagram kommutiert: Nach Konstruktion ist $\phi = \psi \pi$, sowie $F = \tilde{F} \pi$ und $\tilde{F} = \bar{F} \psi$. Daher ist auch
\[
 F = \tilde{F} \pi = \bar{F} \psi \pi = \bar{F} \phi.
\]
Da für alle $s \in S^1$
\[
 \bar{F}(s) = \bar{F} \phi (s,1) = F(s,1) = f(s)
\]
ist $\bar{F}$ eine stetige Fortsetzung von $f$ auf $D^2$.

Angenommen, $f: S^1 \to X$ lässt sich zu einer stetigen Abbildung $\bar{F} : D^2 \to X$ fortsetzen. Mithilfe der stetigen Abbildung
\[
 \phi: S^1 \times [0,1] \to D^2, (s,t) \mapsto ts
\]
erhalten wir dann die Homotopie
\[
 F := \bar{F} \phi : S^1 \times [0,1] \to X.
\]
Da für alle $s \in S^1$
\begin{align*}
 F(s,0) &= \bar{F}(\phi(s,0)) = \bar{F}(0) = \const \text{ und} \\
 F(s,1) &= \bar{F}(\phi(s,1)) = \bar{F}(s) = f(s)
\end{align*}
ist $f$ homotop zu einer konstanten Schlinge.





\section{}
Ist $X$ zusammenziehbar, so gibt es einen einelementigen Raum $\ast$ und eine Homotopieäquivalenz $\varphi : X \to \ast$. Diese induziert, wie aus Aufgabe 4.4 bekannt, für jeden Raum $K$ eine Bijektion
\[
 \varphi^K_* :  [K,X] \to [K,\ast], f \mapsto \varphi f.
\]
Da $[K, \ast]$ immer einelementig ist, ist dann $[K,X]$ für jeden Raum $K$ einelementig.

Besteht $[K,X]$ für jeden Raum $K$ aus nur einem Element, so ist insbesondere $[X,X]$ einelementig, also jede (konstante) Abbildung $X \to X$ homotop zu $\id_X$.

Ist $\id_X$ homotop zu einer konstanten Abbildung, so gibt es ein $x_0 \in X$, so dass $\id_X \simeq f$ für die konstante Funktion $f : X \to X, x \mapsto x_0$. Für den Teilraum $\ast = \{x_0\}$ ist dann für die kanonische Inklusion $\iota : \ast \to X$ und $g : X \to \ast, x \mapsto x_0$
\[
 g \iota = \id_{\ast} \text{ und } \iota g = f \simeq \id_X.
\]
Also ist $g$ eine Homotopieäquivalenz und $X$ deshalb zusammenziehbar.






\section{}


\subsection{}
Für die Homotopie
\[
 F: D^2 \times [0,1], (x,t) = tx
\]
ist für alle $x \in D^2$
\begin{align*}
 F(x,0) &= 0 = \const \text{ und}\\
 F(x,1) &= x = \id_{D^2}(x).
\end{align*}
Deshalb ist $\id_{D^2}$ homotop zu einer konstanten Abbildung. Also ist $D^2$ zusammenziehbar und deshalb $\left[ K,D^2 \right]$ für jeden Raum $K$ einelementig. Insbesondere ist daher $\mc{S}\left(D^2\right) = \left[ S^1, D^2 \right]$ einelementig.

Gibt es eine stetige Abbildung $r : D^2 \to S^1$ mit $r_{|S^1} = \id_{S^1}$, so ergibt sich aus den Funktoreigenschaften von $X \mapsto \mc{S}(X)$ das kommutatives Diagram in Abbildung \ref{fig: SX Funktor}, wobei $\iota : S^1 \to D^2$ die kanonische Inklusion bezeichnet.
\begin{figure}\centering
 \begin{tikzpicture}[node distance = 6em, auto]
  \node (S1 1) {$S^1$};
  \node [right of = S1 1](D2) {$D^2$};
  \node [right of = D2] (S1 2) {$S^1$};
  \node [below of = S1 1](S S1 1) {$\mc{S}(S^1)$};
  \node [below of = D2] (S D2) {$\mc{S}(D^2)$};
  \node [below of = S1 2] (S S1 2) {$\mc{S}(S^2)$};
  \draw[->] (S1 1) to node {$\iota$} (D2);
  \draw[->] (D2) to node {$r$} (S1 2);
  \draw[->] (S1 1) to (S S1 1);
  \draw[->] (D2) to (S D2);
  \draw[->] (S1 2) to (S S1 2);
  \draw[->] (S S1 1) to node[swap] {$\mc{S}(\iota)$}  (S D2);
  \draw[->] (S D2) to node[swap] {$\mc{S}(r)$} (S S1 2);
 \end{tikzpicture}
 \caption{I dunno.}
 \label{fig: SX Funktor}
\end{figure}

Da $r \iota = \id_{S^1}$ muss auch $\mc{S}(r) \mc{S}(\iota) = \id_{\mc{S}\left(S^1\right)}$. Inbesondere muss daher $\mc{S}(r)$ surjektiv sein, da es $\mc{S}(\iota)$ als Rechtsinverses besitzt. Es ist jedoch $\mc{S}(D^2)$ einelementig und $\mc{S}(S^1)$ abzählbar unendlich (aus der Vorlesung bekannt). Dieser Widerspruch zeigt, dass keine solche Abbildung $r$ existieren kann.











\end{document} 
