\documentclass[a4paper,10pt]{article}
%\documentclass[a4paper,10pt]{scrartcl}

\usepackage{xltxtra}
\usepackage{../mystyle}

% die Nummer des Aufgabenblattes
\newcommand{\blattNummer}{1}

\setromanfont[Mapping=tex-text]{Linux Libertine O}
% \setsansfont[Mapping=tex-text]{DejaVu Sans}
% \setmonofont[Mapping=tex-text]{DejaVu Sans Mono}

\title{\sc Einführung in die Geometrie und Topologie \\ \Large Blatt 1}
\author{Jendrik Stelzner}
\date{\today}

\begin{document}
\maketitle





\section{}
Da in der Aufgabenstellung nicht angegeben ist, bezüglich welcher Topologien die Räume betrachtet werden sollen, gehen wir davon aus, dass die durch die euklidische Norm $\|\cdot\|$ induzierte Topologie gemeint ist.


Wir definieren

\[
 f : \inn\left(D^2\right) \to \inn\left(D^2\right),
 x \mapsto
 \begin{pmatrix}
  \cos \frac{1}{1-\|x\|} & -\sin \frac{1}{1-\|x\|} \\
  \sin \frac{1}{1-\|x\|} & \cos \frac{1}{1-\|x\|}
 \end{pmatrix}
 \cdot x.
\]
$f$ ist wohldefiniert, denn
\[
 \inn\left(D^2\right) = \{x \in \R^2 : \|x\| < 1\}.
\]
und $\|f(x)\| = \|x\|$ für alle $x \in \inn\left(D^2\right)$, da Rotationsmatrizen orthogonal sind.

Wir behaupten, dass $f$ ein Homöomorphismus von $\inn\left(D^2\right)$ ist.

Zum Nachweis der Bijektivität definieren wir
\[
 g : \inn\left(D^2\right) \to \inn\left(D^2\right),
 x \mapsto
 \begin{pmatrix}
  \cos \frac{1}{1-\|x\|} & \sin \frac{1}{1-\|x\|} \\
  -\sin \frac{1}{1-\|x\|} & \cos \frac{1}{1-\|x\|}
 \end{pmatrix}
 \cdot x.
\]
Die Wohldefiniertheit von $g$ ergibt sich analog zu der von $f$. Da $\|f(x)\| = \|x\|$ für alle $x \in \inn\left(D^2\right)$ sieht man, dass $(fg) (x) = x$ und $(gf)(x) = x$ für alle $x \in \inn\left(D^2\right)$, also $fg = gf = \id_{\inn(D^2)}$. Das zeigt, dass $f$ bijektiv ist mit $g = f^{-1}$.

Die Stetigkeit von $f$ und $g$ ergibt sich direkt daraus, dass sie Verknüpfung stetiger Funktionen sind. Das zeigt, dass $f$ ein Homöomorphismus ist.

Wir behaupten weiter, dass sich $f$ nicht zu einer stetigen Abbildung $D^2 \to D^2$ fortsetzen lässt. Angenommen, es gebe eine solche stetige Fortsetzung $F$ von $f$. Wir betrachten die Folge $(a_n)_{n \geq 1}$ auf $D^2$ mit
\[
 a_n = \vect{ 1-\frac{1}{n\pi} \\ 0} \text{ für alle } n \geq 1.
\]
Offenbar gilt $a_n \to e_1 = (1,0)^T$ für $n \to \infty$ in $D^2$. Aufgrund der Folgenstetigkeit von $F$ (in metrischen Räumen ist Folgenstetigkeit äquivalent zu Stetigkeit) ist daher auch $F(a_n) \to F(e_1)$ für $n \to \infty$ in $D^2$. Da $a_n \in \inn(D^2)$ für alle $n \geq 1$ ist jedoch
\[
 F(a_n) =
 f(a_n) =
 \begin{pmatrix}
  \cos\left(n\pi\right) & -\sin\left(n\pi\right) \\
  \sin\left(n\pi\right) &  \cos\left(n\pi\right)
 \end{pmatrix}
 a_n
 = (-1)^n a_n \text{ für alle } n \geq 1,
\]
und die Folge $((-1)^n a_n)_{n \geq 1}$ konvergiert in $D^2$ offensichtlich nicht. Dieser Widerspruch zeigt, dass $f$ keine stetige Fortsetzung $D^2 \to D^2$ besitzt.





\section{}


\subsection{}
Angenommen $[0,1]$ ist nicht zusammenhängend. Dann gibt es $U, V \subseteq [0,1]$ mit $U, V \neq \emptyset$ und $[0,1] = U \dotcup V$, so dass $U$ und $V$ offen in $[0,1]$ sind. Dabei können wir o.B.d.A. davon ausgehen, dass $1 \in V$.

Wir bemerken, dass für jede Folge $(a_n)_{n \in \N}$ auf $U$ mit $a_n \to a$ in $\R$ auch $a \in U$ ist: Da $V$ offen in $[0,1]$ ist gibt es $W \subseteq \R$ offen, so dass $V = [0,1] \cap W$. Es ist daher $U = [0,1] \cap (\R-W)$ abgeschlossen in $\R$, und deshalb $a \in U$. (Aus der Analysis ist bereits bekannt, dass eine Menge $A \subseteq \R$ genau dann abgeschlossen ist, wenn für jede Folge $(a_n)_{n \in \N}$ auf $A$ mit $a_n \to a$ in $\R$ auch $a \in A$.) Analog ergibt sich, dass für jede Folge $(b_n)_{n \in \N}$ auf $V$ mit $b_n \to b$ in $\R$ schon $b \in V$.

Sei nun $c = \sup U$. Da $\emptyset \neq U \subseteq [0,1]$ ist $c \in [0,1]$, also entweder $c \in U$ oder $c \in V$. Da es nach Definition von $c$ eine Folge $(a_n)_{n \in \N}$ in $U$ mit $a_n \to c$ in $\R$ gibt, muss $c \in U$. Inbesondere ist daher $c \not\in V$ und $c < 1$.

Da $(c,1] \subseteq V$ gibt es eine Folge $(b_n)_{n \in \N}$ auf $V$ mit $b_n \to c$ in $\R$. Daher muss $c \in V$, was im Widerspruch zu $c \not\in V$ steht.

Das zeigt, dass es keine solchen Mengen $U$ und $V$ geben kann. Also muss $[0,1]$ zusammenhängend sein.


\subsection{}
Sei $X$ ein wegzusammenhängender topologischer Raum. Angenommen, $X$ wäre nicht zusammenhängend. Dann gibt es offene Mengen $U, V \subseteq X$ mit $U, V \neq \emptyset$, so dass $X = U \dotcup V$. Es sei $x \in U$ und $y \in V$.

Da $X$ wegzusammenhängend ist gibt es eine stetige Abbildung $\lambda : [0,1] \to X$ mit $\lambda(0) = x$ und $\lambda(1) = y$. Es ist daher
\[
 [0,1] = \lambda^{-1}(X) = \lambda^{-1}(U \dotcup V) = \lambda^{-1}(U) \dotcup \lambda^{-1}(V),
\]
mit $\lambda^{-1}(U) \neq \emptyset$ da $0 \in \lambda^{-1}(U)$ und $\lambda^{-1}(V) \neq \emptyset$ da $1 \in \lambda^{-1}(V)$. Da $U, V$ offen in $X$ sind und $\lambda$ stetig ist, sind $\lambda^{-1}(U), \lambda^{-1}(V)$ offen in $[0,1]$. Es folgt, dass $[0,1]$ nicht zusammenhängend ist, was falsch ist.

Das zeigt, dass es keine solchen Mengen $U$ und $V$ gibt. Also ist $X$ zusammenhängend.





\addtocounter{section}{1}





\section{}


\subsection{}
Für alle $x,y \in X$ ist
\[
 d'(x,y) = 0 \Leftrightarrow \frac{d(x,y)}{d(x,y)+1} = 0 \Leftrightarrow d(x,y) = 0 \Leftrightarrow x=y
\]
und
\[
 d'(x,y) = \frac{d(x,y)}{d(x,y)+1} = \frac{d(y,x)}{d(y,x)+1} = d'(y,x),
\]
da $d$ eine Metrik auf $X$ ist. Die Dreiecksungleichung für $d'$ ergibt sich aus der Dreiecksungleichung für $d$ durch
\begin{align*}
 d'(x,z)
 &= \frac{d(x,z)}{d(x,z)+1}
 = 1 - \frac{1}{d(x,z)+1}
 \leq 1 - \frac{1}{d(x,y)+d(y,z)+1} \\
 &= \frac{d(x,y)+d(y,z)}{1+d(x,y)+d(y,z)}
 = \frac{d(x,y)}{1+d(x,y)+d(y,z)} + \frac{d(y,z)}{1+d(x,y)+d(y,z)}\\
 &\leq \frac{d(x,y)}{1+d(x,y)} + \frac{d(y,z)}{1+d(y,z)}
 = d'(x,y) + d'(y,z)
\end{align*}
für alle $x,y,z \in X$. Das zeigt, dass $d''$ eine Metrik auf $X$ ist.


\subsection{}
Für alle $x,y \in X$ ist
\[
 d''(x,y) = 0
 \Leftrightarrow \min\{d(x,y),1\} = 0
 \Leftrightarrow d(x,y) = 0
 \Leftrightarrow x=y
\]
und
\[
 d''(x,y) = \min\{d(x,y),1\} = \min\{d(y,x),1\} = d''(y,x),
\]
da $d$ eine Metrik auf $X$ ist. Die Dreiecksungleichung für $d''$ ergibt sich aus der Dreiecksungleichung für $d$ durch
\begin{align*}
 d''(x,z)
 &= \min\{d(x,z),1\}
 \leq \min\{d(x,y)+d(y,z),1\} \\
 &\leq \min\{d(x,y),1\} + \min\{d(y,z),1\}
 = d''(x,y)+d''(y,z)
\end{align*}
für alle $x,y,z \in X$. Dabei haben wir genutzt, dass
\[
 \min\{a+b,c\} \leq \min\{a+b,a+c,c+b,2c\} = \min\{a,c\}+\min\{b,c\}
\]
für alle $a,b,c \geq 0$.


\subsection{}
Es ist klar, dass $d$ und $d''$ die gleich Topologie induzieren, denn für eine Teilmenge $U \subseteq X$ und einen Punkt $x \in U$ gibt es genau dann ein $\varepsilon > 0$ mit $B_\varepsilon(x) \subseteq U$, wenn es ein $0 < \varepsilon’ \leq 1$ mit $B_{\varepsilon'}(x) \subseteq U$ gibt. (Existiert ein solches $\varepsilon'$, so kann man $\varepsilon = \varepsilon'$ wählen; existiert ein solches $\varepsilon$, so kann man $\varepsilon' = \min\{\varepsilon,1\}$ wählen.)

$d'$ und $d''$ induzieren die gleiche Topologie auf $X$, da
\[
 d'(x,y) \leq d''(x,y) \leq 2d'(x,y) \text{ für alle } x,y \in X,
\]
was sich aus
\[
 \frac{a}{a+1} \leq \min\{a,1\} \leq \frac{2a}{a+1} \text{ für alle } a \geq 0.
\]
ergibt.

Der erste Teil der Ungleichung folgt daraus, dass für alle $a \geq 0$
\[
 \frac{a}{a+1} \leq a \text{ und } \frac{a}{a+1} \leq 1
\]
und damit
\[
 \frac{a}{a+1} \leq \min\{a,1\}.
\]
Der zweite Teil der Ungleichung ergibt sich wegen
\[
 \min\{a,1\} \leq \frac{2a}{a+1}
 \Leftrightarrow (a+1)\min\{a,1\} \leq 2a \text{ für alle } a \geq 0
\]
durch eine einfache Fallunterscheidung: Für $0 \leq a < 1$ ist
\[
 (a+1)a \leq 2a \Leftrightarrow a(1-a) \geq 0,
\]
was offenbar gilt, und für $a \geq 1$ ist
\[
 a+1 \leq 2a \Leftrightarrow a \geq 1.
\]
Das zeigt, dass auch $d'$ und $d''$ die gleiche Topologie induzieren.





\end{document}
