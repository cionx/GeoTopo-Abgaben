\documentclass[a4paper,10pt]{article}
%\documentclass[a4paper,10pt]{scrartcl}

\usepackage{xltxtra}
\usepackage{../mystyle}

% die Nummer des Aufgabenblattes
\newcommand{\blattNummer}{2}

\setromanfont[Mapping=tex-text]{Linux Libertine O}
% \setsansfont[Mapping=tex-text]{DejaVu Sans}
% \setmonofont[Mapping=tex-text]{DejaVu Sans Mono}

\title{\sc Einführung in die Geometrie und Topologie \\ \Large Blatt 2}
\author{Jendrik Stelzner}
\date{\today}

\begin{document}
\maketitle





\section{}
\begin{defi}
 Sei $X$ ein topologischer Raum und $x \in X$. Eine Kollektion $\mc{B}$ von Umgebungen von $x$ heißt \emph{Umgebungsbasis} von $x$, falls es für jede Umgebung $N$ von $x$ ein $M \in \mc{B}$ gibt, so dass $M \subseteq N$.
 
 Wir sagen, dass $X$ das \emph{erste Abzählbarkeitsaxiom} erfüllt, bzw. dass $X$ \emph{erstabzählbar} ist, falls es für alle $x \in X$ eine abzählbare Umgebungsbasis $\mc{B}_x$ von $x$ gibt.
\end{defi}


\begin{bem}\label{bem: metrische Räume erstabzählbar}
 Jeder metrische Raum ist erstabzählbar. Für einen metrischen Raum $X$ und $x \in X$ bildet nämlich
 \[
  \mc{B}_x := \{B_\varepsilon(x) : \varepsilon > 0, \varepsilon \in \Q\}
 \]
 offensichtlich eine Umgebungsbasis von $x$.
\end{bem}


\begin{bem}\label{bem: offene Umgebungsbasis}
 Besitzt $x \in X$ eine abzählbare Umgebungsbasis $\mc{B}$, so besitzt $x$ auch eine abzählbare Umgebungsbasis $\mc{U}$ von offenen Umgebungen. Es enthält nämlich jedes $B \in \mc{B}$ eine offene Umgebung $U_B$ von $x$. Es sei dann
 \[
  \mc{U} := \{U_B : B \in B\}.
 \]
 Dass $\mc{U}$ eine Umgebungsbasis von $x$ ist, folgt daraus, dass es für jede Umgebung $N$ von $x$ ein $B \in \mc{B}$ gibt mit $B \subseteq N$, und deshalb
 \[
  U_B \subseteq B \subseteq N.
 \]
\end{bem}



\begin{bem}\label{bem: erstabzählbar homöomorphieinvarinat}
 Sind $X, Y$ topologische Räume und $X \cong Y$, so ist offenbar $X$ genau dann erstabzählbar, wenn $Y$ erstabzählbar ist.
\end{bem}


Die drei Räume $A, B, C$ sind paarweise nicht homöomorph zueinander. Zunächst zeigen wir, dass die hawaiischen Ohrring als einziger der drei Räume kompakt sind.

Der Raum $A$ ist nicht kompakt, da ein Teilraum $X \subseteq \R^n$ genau dann kompakt ist, wenn $X$ in $\R^n$ abgeschlossen und beschränkt ist. Da $A$ offenbar nicht beschränkt ist, ist $A$ nicht kompakt.

Die hawaiischen Ohrringe sind kompakt: Ist $\mc{U}$ eine offene Überdeckung von $B$ in $\R^2$, so gibt es ein $U_0 \in \mc{U}$ mit $0 \in U_0$. Da $U_0$ offen in $\R^2$ ist, gibt es ein $\varepsilon > 0$ mit $B_\varepsilon(0) \subseteq U_0$. Es sei $N \in \N, N \geq 1$, so dass $N > 1/\varepsilon$. Bezeichnet $K_n$ den Kreis mit Mittelpunkt $(0,1/n)$ und Radius $1/n$ für alle $n \geq 1$, so ist also
\[
 B = \bigcup_{n \geq 1} K_n \subseteq U_0 \cup \bigcup_{n=1}^{N} K_n.
\]
Da alle $K_n$ offenbar abgeschlossen und beschränkt sind, also kompakt, und die endliche Vereinigung kompakter Mengen offenbar kompakt ist, besitzt $\mc{U}$ als offene Überdeckung von $\bigcup_{n=1}^N K_n$ eine endliche Teilüberdeckung $\mc{V} \subseteq \mc{U}$ von $\bigcup_{n=1}^N K_n$. Es ist daher $\mc{V} \cup \{U_0\} \subseteq \mc{U}$ eine endliche Teilüberdeckung von $B$. Das zeigt, dass die hawaiischen Ohrringe kompakt ist.

Der Raum $C$ ist nicht kompakt. Bezeichnet $\pi : \R \to \R/\!\sim$ die kanonische Projektion, so ist nach der Definition der Quotiententopologie eine Teilmenge $U \subset C$ genau dann offen, wenn $\pi^{-1}(U)$ offen in $\R$ ist. Für $A \subseteq \R$ mit $A \cap \Z = \emptyset$ oder $A \cap \Z = \Z$ ist $\pi^{-1}(\pi(A)) = A$, also ist $\pi(U)$ offen in $C$ für jede offene Menge $U \subseteq \R$, für die $U \cap \Z = \emptyset$ oder $U \cap \Z = \Z$. Für die offene Überdeckung
\[
 \mc{U} := \left\{ \bigcup_{n \in \Z} B_{1/3}(n) \right\} \cup \bigcup_{n \in \Z} \{(n,n+1)\}
\]
von $\R$ ist daher $\mc{V} := \{\pi(U) : U \in \mc{U}\}$ eine offene Überdeckung von $C$. Da es für alle $n+1/2$ mit $n \in \Z$ eine eindeutige Menge in $\mc{U}$ gibt, die $n+1/2$ enthält, und $\pi$ auf $\R-\Z$ injektiv ist, folgt daraus, dass es für alle $\pi(n+1/2)$ mit $n \in \Z$ eine eindeutige Menge in $\mc{V}$ gibt, die $\pi(n+1/2)$ enthält. Deshalb besitzt $\mc{V}$ keine endliche Teilüberdeckung. Das zeigt, dass $C$ nicht kompakt ist.

Damit haben wir gezeigt, dass die hawaiischen Ohhringe kompakt ist, $A$ und $C$ aber nicht. Also sind die hawaiischen Ohrring zu keinem der anderen beiden Räume homöomorph.

Nach Bemerkung \ref{bem: metrische Räume erstabzählbar} ist $A$ erstabzählbar. Wir zeigen nun noch, dass $C$ nicht erstabzählbar ist, wodurch sich nach Bemerkung \ref{bem: erstabzählbar homöomorphieinvarinat} ergibt, dass auch $A$ und $C$ nicht homöomorph sind.

Angenommen, $C$ ist erstabzählbar. Dann hat $\pi(0) \in C$ nach Bemerkung \ref{bem: offene Umgebungsbasis} eine abzählbare Umgebungsbasis $\mc{U} = \{U_n : n \in \Z\}$ von offenen Umgebungen. Wir schreiben $V_n := \pi^{-1}(U_n)$ für alle $n \in \N$ und setzen
\[
 \mc{V} := \{V_n : n \in \Z\}.
\]
Für alle $n \in \Z$ gilt, dass $\Z \subseteq V_n$, da $\pi(0) \in U_n$, und dass $V_n$ offen ist, da $U_n$ offen ist und $\pi$ stetig. Für alle $n \in \Z$ gibt es daher ein $r_n > 0$ so dass $B_{r_n}(n) \subseteq V_n$. Für alle $n \in \Z$ definieren wir
\[
 r'_n := \min\{r_n/2, 1/3\}
\]
und setzen
\[
 W := \bigcup_{n \in \Z} B_{r'_n}(n).
\]
$W$ ist eine offene Menge mit $\Z \subseteq W$ und $W \subsetneq V_n$ für alle $n \in \Z$, wobei sich $W$ und $V_n$ um je überabzählbar viele Element unterschieden. Daher ist $\pi(W) \subseteq C$ eine offene  Umgebung von $\pi(0) \in C$ mit $\pi(W) \subsetneq U_n$ für alle $n \in \Z$. Dies steht im Widerspruch dazu, dass $\mc{U}$ eine Umgebungsbasis von $\pi(0)$ ist.





\section{}


\addtocounter{subsection}{1}


\subsection{}
Es ist klar, dass $R$ wegzusammenhängend ist und $R - (\{0\} \times [0,4])$ in die beiden Wegzusammenhangskomponenten $[-1,0) \times [0,4]$ und $(0,1] \times [0,4]$ zerfällt. Da $\pi_0$ funktoriell ist, zerfällt daher auch $M-K$ in höchstens zwei Wegzusammenhangskomponenten, wobei $q([-1,0) \times [0,4])$ und $q((0,1] \times [0,4])$ wegzu\-sammen\-hängend in $M-K$ sind. Wir zeigen, dass $M-K$ bereits wegzusammenhängend ist. Da Wegzusammenhangskomponenten entweder disjunkt oder gleich sind, und jede wegzusammenhängende Teilmenge in einer Wegzusammenhangskomponente liegt, reicht es hierfür zu zeigen, dass $q(X)$ wegzusammenhängend ist für
\[
 X := \{-1\} \times [0,4] \cup \{1\} \times [0,4].
\]
Dies zeigen wir, indem wir zeigen, dass $q(X) \cong S^1$. ($S^1$ ist als Quotenten des wegzusammenhängenden Raumes $[0,1]$ ebenfalls wegzusammenhängend.)

Wir betrachten die Abbildung $f : X \to S^1$ mit
\[
f(s,t)
=
\begin{cases}
   \left(\cos\left(\frac{\pi}{4}t\right),\sin\left(\frac{\pi}{4}t\right)\right) & \text{falls } s = 1, \\
 - \left(\cos\left(\frac{\pi}{4}t\right),\sin\left(\frac{\pi}{4}t\right)\right) & \text{falls } s = -1,
\end{cases}
=  s \left(\cos\left(\frac{\pi}{4}t\right),\sin\left(\frac{\pi}{4}t\right)\right).
\]
Es ist klar, dass $f$ surjektiv ist und als Bijektion $\tilde{f}$ über $q(X)$ faktorisiert. Wir erhalten ein entsprechendes kommutative Diagram.

\begin{center}
 \begin{tikzpicture}[node distance = 7em,auto]
  \node (X) {$X$};
  \node (qX) [below left of = X] {$q(X)$};
  \node (S1) [below right of = X] {$S^1$};
  \draw[->] (X) to node [swap] {$q$} (qX);
  \draw[->] (X) to node {$f$} (S1);
  \draw[->] (qX) to node [swap] {$\tilde{f}$} (S1);
 \end{tikzpicture}
\end{center}

Wir bemerken, dass $\tilde{f}$ stetig ist: Wir zeigen, dass $\tilde{f}^{-1}(B_\varepsilon(x))$ in $q(X)$ offen ist für alle $x \in S^1$ und $\varepsilon > 0$, wobei wir zur einfacheren Notation
\[
 B_\varepsilon(x) = \{y \in S^1 : \|x-y\| < \varepsilon\} \subseteq S^1
\]
verstehen wollen. ($\|\cdot\|$ bezeichnet die übliche Norm auf $\R^2$.) Da diese $\varepsilon$-Bälle eine Basis der Topologie von $S^1$ bilden, zeigen wir damit die Stetigkeit von $\tilde{f}$. Wegen der Definition der Quotientenraumtopologie genügt es hierfür zu zeigen, dass $q^{-1}(\tilde{f}^{-1}(B_\varepsilon(x)))$ offen in $X$ ist. Das Urbild eines solchen $\varepsilon$-Balles hat für passende $u, t \in (0,4)$ die Form
\begin{align*}
 &\{1\} \times (u,t) \text{ oder } \\
 &\{-1\} \times (u,t) \text{ oder } \\
 &\{-1\} \times [0,u) \cup \{1\} \times (t,4] \text{ oder } \\
 &\{-1\} \times (t,4] \cup \{1\} \times [0,u).
\end{align*}
Da all diese Mengen offen in $X$ sind, zeigt dies die Stetigkeit von $\tilde{f}$.

Damit ist $\tilde{f}$ eine stetige Bijektion. Da $S^1$ hausdorff ist und $q(X)$ als Quotient des kompakten Raumes $X$ ebenfalls quasikompakt ist, ist $\tilde{f}$ schon ein Homöomorphismus. Das zeigt, dass $q(X) \cong S^1$ und damit, dass $M-K$ wegzusammenhängend ist.










\end{document}
