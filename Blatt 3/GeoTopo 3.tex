\documentclass[a4paper,10pt]{article}
%\documentclass[a4paper,10pt]{scrartcl}

\usepackage{xltxtra}
\usepackage{../mystyle}

% die Nummer des Aufgabenblattes
\newcommand{\blattNummer}{3}

\setromanfont[Mapping=tex-text]{Linux Libertine O}
% \setsansfont[Mapping=tex-text]{DejaVu Sans}
% \setmonofont[Mapping=tex-text]{DejaVu Sans Mono}

\title{\sc Einführung in die Geometrie und Topologie \\ \Large Blatt 3}
\author{Jendrik Stelzner}
\date{\today}

\begin{document}
\maketitle





\section{}
Es sei $X$ ein quasikompakter Raum. Wir nehmen an, dass $X$ nicht folgenkompakt ist. Dann gibt es eine Folge $(x_n)_{n \in \N}$ auf $X$, die keinen Häufungspunkt besitzt.

Es gibt also für jedes $x \in X$ eine Umgebung $U_x$ von $x$, so dass nur endlich viele Folgeglieder in $U_x$ sind. Da $U_x$ eine offene Umgebung von $x$ enthält, können wir o.B.d.A. davon ausgehen, dass die $U_x$ alle offen sind.

Es ist $\{U_x : x \in X\}$ eine offene Überdeckung von $X$. Da $X$ quasikompakt ist gibt es daher $x_1, \ldots, x_n \in X$ mit $X = U_{x_1} \cup \ldots \cup U_{x_n}$. Da $U_{x_1}, \ldots, U_{x_n}$ jeweils nur endlich viele Folgeglieder enthalten, enthält $X$ nur endlich viele Folgeglieder — ein offensichtlicher Widerspruch.

Das zeigt, dass $(x_n)_{n \in \N}$ einen Häufungspunkt besitzt, und damit, dass $X$ folgenkompakt ist. Also ist jeder quasikompakte Raum folgenkompakt.





\section{}
Wir setzen $Y := X \times X$ und $\Delta := \Delta(X)$. Für $A,B \subseteq X$ ist
\begin{align*}
                  (A \times B) \cap \Delta \neq \emptyset
 &\Leftrightarrow \exists x \in X : (x,x) \in A \times B \\
 &\Leftrightarrow \exists x \in X : x \in A \text{ und } x \in B \\
 &\Leftrightarrow A \cap B \neq \emptyset,
\end{align*}
also
\[
 A \times B \subseteq Y-\Delta
 \Leftrightarrow (A \times B) \cap \Delta = \emptyset
 \Leftrightarrow A \cap B = \emptyset.
\]

Da die Mengen der Form $U \times V \subseteq Y$ mit offenen $U, V \subseteq X$ eine Basis der Produkttopologie auf $Y$ bilden, gilt
\begin{align*}
                &\, W \subseteq Y \text{ ist offen} \\
 \Leftrightarrow&\, \forall (x,y) \in W \exists U,V \subseteq X \text{ offen mit } (x,y) \in U \times V \subseteq W \\
 \Leftrightarrow&\, \forall (x,y) \in W \exists U,V \subseteq X \text{ offen mit } x \in U, y \in V \text{ und } U \times V \subseteq W.
\end{align*}

Zusammengefasst gilt daher
\begin{align*}
                &\, \Delta \text{ ist abgeschlossen in } Y \\
 \Leftrightarrow&\, Y-\Delta \text{ ist offen in } Y \\
 \Leftrightarrow&\, \forall (x,y) \in (Y-\Delta) \exists U,V \subseteq X \text{ offen mit } x \in U, y \in V, U \times V \subseteq Y-\Delta \\
 \Leftrightarrow&\, \forall (x,y) \in (Y-\Delta) \exists U,V \subseteq X \text{ offen mit } x \in U, y \in V, U \cap V = \emptyset \\
 \Leftrightarrow&\, \forall x,y \in X \text{ mit } x \neq y \exists U,V \subseteq X \text{ offen mit } x \in U, y \in V, U \cap V = \emptyset \\
 \Leftrightarrow&\, X \text{ ist Hausdorffsch.}
\end{align*}

Fun fact: Ein alternativer, intuitiverer Beweis lässt sich mithilfe von Netzen formulieren: Dass $\Delta$ abgeschlossen ist, ist äquivalent dazu, dass für jedes Netz $(h_\alpha)$ auf $\Delta$, das gegen ein $h \in X \times X$ konvergiert, bereits $h \in \Delta$. Da $(h_\alpha) = (x_\alpha, x_\alpha)$ für ein Netz $x_\alpha$ auf $X$ und $h = (x,y)$ mit $x,y \in X$ ist dies äquivalent dazu, dass für jedes Netz $(x_\alpha)$ auf $X$ mit $x_\alpha \to x$ und $x_\alpha \to y$ bereits $x = y$. Dies bedeudet gerade, dass Grenzwerte von Netzen auf $X$ eindeutig sind, was bekanntermaßen äquivalent dazu ist, dass $X$ Hausdorff ist.
















\end{document}
