\documentclass[a4paper,10pt]{article}
%\documentclass[a4paper,10pt]{scrartcl}

\usepackage{xltxtra}
\usepackage{../mystyle}

% die Nummer des Aufgabenblattes
\newcommand{\blattNummer}{3}

\setromanfont[Mapping=tex-text]{Linux Libertine O}
% \setsansfont[Mapping=tex-text]{DejaVu Sans}
% \setmonofont[Mapping=tex-text]{DejaVu Sans Mono}

\title{\sc Einführung in die Geometrie und Topologie \\ \Large Blatt 3}
\author{Jendrik Stelzner}
\date{\today}

\begin{document}
\maketitle





\section{}
Es sei $X$ ein quasikompakter Raum. Wir nehmen an, dass $X$ nicht folgenkompakt ist. Dann gibt es eine Folge $(x_n)_{n \in \N}$ auf $X$, die keinen Häufungspunkt besitzt.

Es gibt also für jedes $x \in X$ eine Umgebung $U_x$ von $x$, so dass nur endlich viele Folgeglieder in $U_x$ sind. Da $U_x$ eine offene Umgebung von $x$ enthält, können wir o.B.d.A. davon ausgehen, dass die $U_x$ alle offen sind.

Es ist $\{U_x : x \in X\}$ eine offene Überdeckung von $X$. Da $X$ quasikompakt ist gibt es daher $x_1, \ldots, x_n \in X$ mit $X = U_{x_1} \cup \ldots \cup U_{x_n}$. Da $U_{x_1}, \ldots, U_{x_n}$ jeweils nur endlich viele Folgeglieder enthalten, enthält $X$ nur endlich viele Folgeglieder — ein offensichtlicher Widerspruch.

Das zeigt, dass $(x_n)_{n \in \N}$ einen Häufungspunkt besitzt, und damit, dass $X$ folgenkompakt ist. Also ist jeder quasikompakte Raum folgenkompakt.




\end{document}
