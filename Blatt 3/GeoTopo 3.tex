\documentclass[a4paper,10pt]{article}
%\documentclass[a4paper,10pt]{scrartcl}

\usepackage{xltxtra}
\usepackage{../mystyle}

% die Nummer des Aufgabenblattes
\newcommand{\blattNummer}{3}

\setromanfont[Mapping=tex-text]{Linux Libertine O}
% \setsansfont[Mapping=tex-text]{DejaVu Sans}
% \setmonofont[Mapping=tex-text]{DejaVu Sans Mono}

\title{\sc Einführung in die Geometrie und Topologie \\ \Large Blatt 3}
\author{Jendrik Stelzner}
\date{\today}

\begin{document}
\maketitle





\section{}
Es sei $X$ ein quasikompakter Raum. Wir nehmen an, dass $X$ nicht folgenkompakt ist. Dann gibt es eine Folge $(x_n)_{n \in \N}$ auf $X$, die keinen Häufungspunkt besitzt.

Es gibt also für jedes $x \in X$ eine Umgebung $U_x$ von $x$, so dass nur endlich viele Folgeglieder in $U_x$ sind. Da $U_x$ eine offene Umgebung von $x$ enthält, können wir o.B.d.A. davon ausgehen, dass alle $U_x$ offen sind.

Es ist $\{U_x : x \in X\}$ eine offene Überdeckung von $X$. Da $X$ quasikompakt ist gibt es daher $y_1, \ldots, y_n \in X$ mit $X = U_{y_1} \cup \ldots \cup U_{y_n}$. Da $U_{y_1}, \ldots, U_{y_n}$ jeweils nur endlich viele Folgeglieder enthalten, enthält $X$ nur endlich viele Folgeglieder — ein offensichtlicher Widerspruch.

Das zeigt, dass $(x_n)_{n \in \N}$ einen Häufungspunkt besitzt, und damit, dass $X$ folgenkompakt ist. Also ist jeder quasikompakte Raum folgenkompakt.





\section{}
Wir setzen $Y := X \times X$ und $\Delta := \Delta(X)$. Für $A,B \subseteq X$ ist
\begin{align*}
                  (A \times B) \cap \Delta \neq \emptyset
 &\Leftrightarrow \exists x \in X : (x,x) \in A \times B \\
 &\Leftrightarrow \exists x \in X : x \in A \text{ und } x \in B \\
 &\Leftrightarrow A \cap B \neq \emptyset,
\end{align*}
und deshalb
\[
 A \times B \subseteq Y-\Delta
 \Leftrightarrow (A \times B) \cap \Delta = \emptyset
 \Leftrightarrow A \cap B = \emptyset.
\]

Da die Mengen der Form $U \times V \subseteq Y$ mit offenen $U, V \subseteq X$ eine Basis der Produkttopologie auf $Y$ bilden, gilt
\begin{align*}
                &\, W \subseteq Y \text{ ist offen} \\
 \Leftrightarrow&\, \forall (x,y) \in W \text{ gibt es } U,V \subseteq X \text{ offen mit } (x,y) \in U \times V \subseteq W \\
 \Leftrightarrow&\, \forall (x,y) \in W \text{ gibt es } U,V \subseteq X \text{ offen mit } x \in U, y \in V \text{ und } U \times V \subseteq W.
\end{align*}

Zusammengefasst gilt daher
\begin{align*}
                &\, \Delta \text{ ist abgeschlossen in } Y \\
 \Leftrightarrow&\, Y-\Delta \text{ ist offen in } Y \\
 \Leftrightarrow&\, \forall (x,y) \in (Y-\Delta) \exists U,V \subseteq X \text{ offen mit } x \in U, y \in V, U \times V \subseteq Y-\Delta \\
 \Leftrightarrow&\, \forall (x,y) \in (Y-\Delta) \exists U,V \subseteq X \text{ offen mit } x \in U, y \in V, U \cap V = \emptyset \\
 \Leftrightarrow&\, \forall x,y \in X \text{ mit } x \neq y \exists U,V \subseteq X \text{ offen mit } x \in U, y \in V, U \cap V = \emptyset \\
 \Leftrightarrow&\, X \text{ ist Hausdorff.}
\end{align*}

Fun fact: Ein alternativer, intuitiverer Beweis lässt sich mithilfe von Netzen formulieren: Es ist
\[
 \overline{\Delta} = \{h \in X \times X : \text{ es gibt ein Netz $(h_\alpha)$ auf $\Delta$ mit $h_\alpha \to h$ in $X \times X$}.
\]
Dass $\Delta$ abgeschlossen ist, ist daher äquivalent dazu, dass für jedes Netz $(h_\alpha)$ auf $\Delta$, das gegen ein $h \in X \times X$ konvergiert, bereits $h \in \Delta$.

Wir bemerken, dass ein Netz $(h_\alpha)$ auf $\Delta$ von der Form
\[
 (h_\alpha) = (x_\alpha, x_\alpha)
\]
ist, wobei $(x_\alpha)$ ein Netz auf $X$ ist, und dass $h = (x,y)$ mit $x,y \in X$. Da ein Netz in einem Produktraum genau dann konvergiert, wenn es in jeder einzelnen Koordinate konvergiert, ist die obige Aussage äquivalent dazu, dass für jedes Netz $(x_\alpha)$ auf $X$ und alle $x,y \in X$ mit $x_\alpha \to x$ und $x_\alpha \to y$ bereits $x = y$.

Es ist also $\Delta$ genau dann abgeschlossen in $X \times X$, wenn Grenzwerte von Netzen auf $X$ eindeutig sind. Dies ist bekanntermaßen äquivalent dazu, dass $X$ Hausdorff ist.





\section{}
Es bezeichne $\sim_n$ die Äquivalenzrelation auf $\R^{n+1} - \{0\}$ mit
\[
 x \sim_n y \Leftrightarrow \exists \lambda \in \R^\times \text{ mit } y = \lambda x.
\]
Außerdem sei
\[
 S^n = \{x \in \R^{n+1} : \|x\|=1\}
\]
die $n$-Sphäre und
\[
 \tilde{D}^n := \{(x_1, \ldots, x_{n+1}) \in S^n : x_{n+1} \geq 0\}.
\]
die „nördliche“ Hemisphäre von $S^n$. Man bemerke, dass $\sim$ auf $S^n$ genau die Antipodenpunkte miteinander identifiziert, also den Punkt $x \in S^n$ mit dem Punkt $-x \in S^n$.
Es bezeichne außerdem $\sim^*_n$ die Äquivalenzrelation auf $D^n$, die die Antipodenpunkte des Randes von $D^n$ miteinander identifziert, also jeden Punkt $x \in S^{n-1} \subseteq D^n$ mit dem Antipodenpunkt $-x \in S^{n-1} \subseteq D^n$. Es seien
\[
 \varphi_n : D^n \to \tilde{D}^n, (x_1, \ldots, x_n) \mapsto \left(x_1, \ldots, x_n, \sqrt{1-\sum_{i=1}^n x_i^2}\right)
\]
und
\[
 \psi_n : S^n/{\sim_n} \to \R P^n, [x]_{\sim_n} \mapsto [x]_{\sim_n}.
\]

Wir haben bereits letze Woche gezeigt, dass
\[
 D^n/{\sim^*_n}
 \cong \tilde{D}^n/{\sim_n}
 \cong S^n/{\sim_n}
 \cong (\R^{n+1}-\{0\})/{\sim_n}
 \cong \R P^n,
\]
wobei der Homöomorphismus $D^n/{\sim^*_n} \cong \tilde{D}^n/{\sim_n}$ durch $\varphi_n$ induziert wird, der Homöomorphismus $\tilde{D}^n/{\sim_n} \cong S^n/{\sim_n}$ durch die kanonische Inklusion $\tilde{D}^n \incl S^n$ induziert wird, und der Homöomorphismus $S^n/{\sim_n} \cong (\R^{n+1}-\{0\})/{\sim_n}$ durch $\psi$ gegeben ist, also durch die kanonische Inklusion $S^n \incl \R^{n+1} - \{0\}$ induziert wird. Auch haben wir im Laufe des Nachweises dieser Homöomorphien gezeigt, dass $S^n/{\sim_n}$ Hausdorff ist.


\subsection{}
Dies haben wir bereits letzte Woche gezeigt.


\subsection{}
Dies folgt direkt daraus, dass $S^n /{\sim_n} \cong \R P^n$, und dass $S^n/{\sim_n}$ Hausdorff ist.


\subsection{}
Es ist offenbar $\sim$ die Einschränkung von $\sim_2$ auf $S^2$. Es bezeichne $\pi : S^2 \to S^2/{\sim}$ die kanonische Projektion.
Wir setzen
\begin{align*}
 S &:= \left\{(x,y,z) \in S^2 : z \geq \frac{1}{2} \text{ oder } z \leq -\frac{1}{2} \right\}, \\
 T &:= \left\{(x,y,z) \in S^2 : -\frac{1}{2} \leq z \leq \frac{1}{2}\right\} \text{ und} \\
 R &:= S \cap T = \left\{(x,y,z) \in S^2 : z = -\frac{1}{2} \text{ oder } z = \frac{1}{2} \right\}.
\end{align*}
Es ist klar, dass $S,T$ und $R$ abgeschlossen in $S^2$ sind, und dass sie saturiert bezüglich $\sim$ sind, dass also $\pi^{-1}(\pi(X)) = X$ für alle $X \in \{S,T,R\}$. Insbesondere sind daher auch $\pi(S), \pi(T)$ und $\pi(R)$ abgeschlossen in $S^2/{\sim}$. (Denn die abgeschlossenen Mengen in $S^2/{\sim}$ sind genau die Bilder von abgeschlossenen, saturierten Teilmengen von $S^2$ unter $\pi$.)

Bekanntermaßen ist $S^2 /{\sim} \cong \R P^2$. Es sei $f : S^2 /{\sim} \to \R P^2$ ein Homöomorphismus (man kann etwa $f$ als $\psi_2$ wählen). Wir setzen
\[
 A := f(\pi(S)), B := f(\pi(T)) \text{ und } C := A \cap B.
\]
Wir bemerken dabei direkt, dass $C =  f(\pi(R))$, da
\begin{align*}
 C
 &= A \cap B
 = f(\pi(S)) \cap f(\pi(T))
 = f(\pi(S) \cap \pi(T)) \\
 &= f(\pi(\pi^{-1}(\pi(S) \cap \pi(T))))
 = f(\pi(\pi^{-1}(\pi(S)) \cap \pi^{-1}(\pi(T)))) \\
 &= f(\pi(S \cap T))
 = f(\pi(R)).
\end{align*}
Da $\pi(X)$ für alle $X \in \{S,T,R\}$ abgeschlossen in $S^2/{\sim}$ ist, und $f$ ein Homöomorphismus ist, sind $A, B$ und $C$ abgeschlossen in $\R P^2$.

Da $f : S^2/{\sim} \to \R P^n$ ein Homöomorphismus ist, ist klar, dass auch die Einschränkung
\[
 \pi(S) \to f(\pi(S)) = A, x \mapsto f(x)
\]
ein Homöomorphismus ist. Daher ist $\pi(S) \cong A$.

Wir bemerken, dass $S/{\sim} \cong \pi(S)$: Die stetige Abbildung $S \to \pi(S), x \mapsto \pi(x)$ faktorisiert offenbar als Bijektion $S/{\sim} \to \pi(S)$ (denn es ist $\pi(s) = \pi(s') \Leftrightarrow s \sim s'$), die nach der universellen Eigenschaft des Quotienten stetig ist (da $\pi$ stetig ist). Da $S$ kompakt ist (denn $S^2$ ist kompakt und $S \subseteq S^2$ abgeschlossen) und $\pi(S)$ Hausdorff ist (denn $\pi(S) \subseteq S^2/{\sim}$ mit $S^2/{\sim}$ Hausdorff) ist g bereits ein Homöomorphismus. Dies lässt sich in dem kommutativen Diagram von Abbildung \ref{fig: S/sim pi(S)} zusammenfassen.
\begin{figure}\centering
 \begin{tikzpicture}[node distance = 7em, auto]
  \node (S) {$S$};
  \node [below left of = S]  (Ssim) {$S/{\sim}$};
  \node [below right of = S] (piS) {$\pi(S)$};
  \draw[->] (S) to node [swap] {$\operatorname{can.}$} (Ssim);
  \draw[->] (S) to node {$\pi$} (piS);
  \draw[->] (Ssim) to node [swap] {$g$}(piS);
 \end{tikzpicture}
 \caption{Die Homöomorphie von $S/{\sim}$ und $\pi(S)$.}
 \label{fig: S/sim pi(S)}
\end{figure}

Das zeigt, dass
\[
 S/{\sim} \cong \pi(S) \cong A.
\]
Komplett analog ergibt sich, dass auch $T/{\sim} \cong B$ und $R/{\sim} \cong C$. Wir zeigen nun, dass $S/{\sim} \cong D^2$, $T/{\sim} \cong M$ und $R/{\sim} \cong S^1$. Dabei ist die Homöomorphie $T/{\sim} \cong M$ bereits aus Aufgabe 2.3 bekannt.

Für die Homöomorphie $S/{\sim} \cong D^2$ betrachten wir die Abbildung
\[
 h : S \to D^2 \text{ mit }
 h(x,y,z) =
 \begin{cases}
   \frac{\sqrt{3}}{2}(x,y) & \text{ falls } z > 0, \\
  -\frac{\sqrt{3}}{2}(x,y) & \text{ falls } z < 0.
 \end{cases}
\]
Diese ist offenbar wohldefiniert und surjektiv. (Man beachte, dass für alle $(x,y,z) \in S$
\[
 \|(x,y)\| = \sqrt{x^2+y^2} = \sqrt{1-z^2} \leq \sqrt{1-\frac{1}{4}} = \frac{\sqrt{3}}{2}
\]
und Gleichheit für $z = \pm 1/2$ gilt.) Es ist auch klar, dass $h$ stetig ist ($h$ ist auf den beiden Zusammenhangskomponenten, in die $S$ zerfällt, offenbar stetig, und somit auch auf ganz $S$). Offenbar faktorisiert $h$ als Bijektion $\tilde{h} : S/{\sim} \to D^2$, die nach der universellen Eigenschaft des Quotienten stetig ist. Da $S/{\sim}$ als Quotient eines quasikompakten Raumes selbser quasi-kompakt ist, und $D^2$ Hausdorff ist, ist $\tilde{h}$ bereits ein Homöomorphismus.

Wir haben gezeigt, dass
\[
 A \cong \pi(S) \cong S/{\sim} \cong D^2.
\]
Komplett analog zeigt man auch dass $R/{\sim} \cong S^1$, und damit, dass
\[
 C \cong \pi(R) \cong R/{\sim} \cong S^1.
\]


\section{}


\subsection{}
Wir zeigen zunächst, dass jedes Element von $\SO(3)$ einen Eigenvektor zum Eigenwert $1$ besitzt.

Sei zunächst $A \in \SO(2)$. Wir behaupten, dass es ein $\varphi \in [-\pi,\pi]$ gibt, so dass
\[
 A = \vect{\cos \varphi & -\sin \varphi \\ \sin \varphi & \cos \varphi}.
\]
Es sei
\[
 A = \vect{a & b \\ c & d}.
\]
Da
\[
 \vect{a & c \\ b & d} = A^T = A^{-1} = \vect{d & -b \\ -c & a}
\]
muss $a = d$ und $b = -c$, also
\[
 A= \vect{a & -c \\ c & a}.
\]
Da
\[
 1 = \det A = a^2 + c^2
\]
gibt es also ein $\varphi \in [-\pi,\pi]$ mit $a = \cos \varphi$ und $c = \sin \varphi$. Das zeigt die Behauptung.

Für $B \in \On(2)$ mit $B \not\in \SO(2)$ ist
\[
 \vect{0 & 1 \\ 1 & 0} B \in \SO(2),
\]
also gibt es ein $\varphi \in [-\pi,\pi]$ mit
\[
 B
 = \vect{0 & 1 \\ 1 & 0} \vect{0 & 1 \\ 1 & 0} B
 = \vect{0 & 1 \\ 1 & 0} \vect{\cos \varphi & -\sin \varphi \\ \sin \varphi & \cos \varphi}
 = \vect{\sin \varphi & \cos \varphi \\ \cos \varphi & -\sin \varphi}.
\]
Für das charakteristische Polynom $\chi_B$ von $B$ gilt daher
\begin{align*}
 \chi_B
 &= (-t+\sin \varphi)(-t-\sin \varphi)-\cos^2 \varphi \\
 &= t^2 - \sin^2 \varphi - \cos^2 \varphi
 = t^2 - 1 \\
 &= (t+1)(t-1).
\end{align*}
Also besitzt $B$ einen Eigenvektor zum Eigenwert $1$. Das zeigt, dass jedes Matrix $B \in \On(2) - \SO(2)$ einen Eigenvektor zum Eigenwert $1$ besitzt. (Diese Aussage ist auch geometrisch klar.)

Sei nun $C \in \SO(3)$. Da $\chi_C$ ein reelles Polynom ungeraden Grades ist, besitzt $\chi_C$ eine reelle Nullstelle, d.h. $C$ besitzt einen reellen Eigenwert $\lambda \in \R$. Da für einen entsprechenden Eigenvektor $v$ gilt, dass
\[
 \|v\| = \|Cv\| = \|\lambda v \| = |\lambda| \|v\|,
\]
muss $|\lambda| = 1$, also $\lambda = 1$ oder $\lambda = -1$. Angenommen es ist $\lambda = -1$. Indem wir das orthogonale Komplement von $\gen{v}_\R$ in $\R^3$ betrachtent, also $\gen{v}_R^\perp$, können wir $C$ durch einen passenden Basiswechsel eine Matrix der Form
\[
 \vect{-1 & \\ & A}
\]
überführen, wobei $A \in \On(2)$. Da $1 = \det C = -1 \det A$ muss $\det A = -1$, also $A \in \On(2) - \SO(2)$. Nach der obigen Diskussion besitzt daher $A$ einen Eigenvektor zum Eigenwert $1$, also auch $C$.

Insbesondere lässt sich jede Matrix $A \in \SO(3)$ durch einen passenden Basiswechsel in die Form
\[
 \vect{1 & 0 \\ 0 & A'}
\]
mit $A' \in \SO(2)$ überführen, also in die Form
\[
 \begin{pmatrix}
  1 &             0 &             0 \\
  0 &  \cos \varphi & -\sin \varphi \\
  0 & -\sin \varphi &  \cos \varphi \\
 \end{pmatrix}
\]
mit $\varphi \in [-\pi,\pi]$, wobei $A$ auf $\gen{v}_\R^\perp$ durch Rotation um den Winkel $\varphi$ operiert, wobei $v$ ein Eigenvektor von $A$ zum Eigenwert $1$ ist.

Sei nun $f \in \SO(3)$ mit Eigenvektor $v$ zum Eigenwert $1$, so dass $f$ auf $\gen{v}_\R^\perp$ durch Rotation um den Winkel $\varphi \in [-\pi,\pi]$ wirkt. Wir können o.B.d.A. davon ausgehen, dass $v \in S^2 \subseteq D^3$. Es hat dann $f$ unter $p$ offensichtlich das Urbild $(\varphi/\pi)v$.


\addtocounter{subsection}{1}
\subsection{}
Der Fall $p(x) = p(y) = E$ ist trivial, es müssen dann $x = y = 0$. Wir betrachten daher im Folgenden den Fall, dass $p(x) = p(y) \neq E$, also insbesonder $x,y \neq 0$.

Da $p(x) = p(y)$, haben $p(x)$ und $p(y)$ die gleiche Rotationsachso, nämlich $\gen{x}_\R$, bzw. $\gen{y}_\R$. Daher müssen $x$ und $y$ linear abhängig sein.

Ist $y = \lambda x$ für ein $\lambda > 0$, so rotieren $p(x)$ und $p(y)$ gleichorientiert, also um den Winkel $\pi \cdot \|x\|$ und $\pi \cdot \|y\|$. Es muss daher $\|x\| = \|y\| = \lambda \|x\|$, also $\lambda = 1$ und somit $x = y$.

Ist $y = \lambda x$ für ein $\lambda < 0$, so rotieren $p(x)$ und $p(y)$ unterschiedlich orientiert, also um die Winkel $\pi \cdot \|x\|$ und $\pi \cdot \|y\|$. Damit diese Rotationen gleich sind, muss $\|x\| = \|y\| = 1$. Es ist dann $\lambda = -1$ und somit $x = -y$.


\addtocounter{subsection}{-1}
\subsection{}
Wir betrachten die Äquivalenzrelation $\sim$ auf $D^3$, die jeden Punkt $x \in S^2 \subseteq D^3$ mit seinem Antipodenpunkt $-x$ identifiziert. Durch den vorherigen Aufgabenteil ergibt sich, dass $x \sim y \Leftrightarrow p(x) = p(y)$ für alle $x,y \in D^3$. Daher faktorisiert $p$ als Bijektion $\tilde{p} : D^3/{\sim} \to \SO(3)$, die nach der universellen Eigenschaft des Quotienten stetig ist (denn $p$ ist nach Annahme stetig). Da $D^3/{\sim}$ als Quotient eines quasikompakten Raumes ebenfalls quasikompakt ist, und $\SO(3) \subseteq R^9$ Hausdorff ist, ist $\tilde{p}$ bereits ein Homöomorphismus.

Wir betrachten das kommutative Diagramm in Abbildung \ref{fig: p Quotientenabbildung}.
\begin{figure}\centering
 \begin{tikzpicture}[node distance = 6em, auto]
  \node (D3sim) {$D^3$};
  \node (D3) [above left of = D3sim] {$D^3$};
  \node (SO3) [above right of = D3sim] {$\SO(3)$};
  \draw[->] (D3) to node {$p$} (SO3);
  \draw[->] (D3) to node [swap] {$\pi$} (D3sim);
  \draw[->] (D3sim) to node [swap] {$\tilde{p}$} (SO3);
 \end{tikzpicture}
 \caption{$p$ ist eine Quotientenabbildung.}
 \label{fig: p Quotientenabbildung}
\end{figure}
Da $\tilde{p}$ und $\pi$ offenbar Quotientenabbildungen sind, ist es offenbar auch $p = \tilde{p} \circ \pi$, denn für $U \subseteq \SO(3)$ ist
\[
 U \text{ ist offen }
 \Leftrightarrow \tilde{p}^{-1}(U) \text{ ist offen }
 \Leftrightarrow \pi^{-1}(U) = \pi^{-1}\left(\tilde{p}^{-1}(U)\right) \text{ ist offen}.
\]


\subsection{}
Wir haben letzte Woche gezeigt, dass das Diagramm in Abbildung \ref{fig: q Quotientenabbildung} kommutiert, und dass $\tilde{q}$ ein Homöomorphismus ist. Da $\pi$ eine Quotientenabbildung ist, ist damit analog zur obigen Argumentation auch $q$ eine Quotientenabbildung. ($\tilde{q}$ entspricht dem Homöomorphismus $D^n/{\sim^*} \cong \R P^n$).
\begin{figure}\centering
 \begin{tikzpicture}[node distance = 6em, auto]
  \node (D3sim) {$D^3$};
  \node (D3) [above left of = D3sim] {$D^3$};
  \node (RP3) [above right of = D3sim] {$\R P(3)$};
  \draw[->] (D3) to node {$q$} (RP3);
  \draw[->] (D3) to node [swap] {$\pi$} (D3sim);
  \draw[->] (D3sim) to node [swap] {$\tilde{q}$} (RP3);
 \end{tikzpicture}
 \caption{$q$ ist eine Quotientenabbildung.}
 \label{fig: q Quotientenabbildung}
\end{figure}


\subsection{}
Es ist $q(x) = q(y)$ genau dann, wenn $\pi(x) = \pi(y)$, also wenn $x \sim^* y$, also wenn $x = y$ oder $x,y \in S^2 \subseteq D^3$ mit $x = -y$.


\subsection{}
Wir haben das kommutative Diagramm wie in Abbildung \ref{fig: alles zusammen}.
\begin{figure}\centering
 \begin{tikzpicture}[node distance = 6em, auto]
  \node (D3) {$D^3$};
  \node [below of = D3] (D3sim) {$D^3/{\sim}$};
  \node [left of = D3sim, node distance = 7em] (SO3) {$\SO(3)$};
  \node [right of = D3sim, node distance = 7em](RP3) {$\R P^3$};
  \draw[->] (D3) to node {$\pi$} (D3sim);
  \draw[->] (D3) to node [swap] {$p$} (SO3);
  \draw[->] (D3) to node {$q$} (RP3);
  \draw[->] (D3sim) to node {$\tilde{p}$} (SO3);
  \draw[->] (D3sim) to node [swap] {$\tilde{q}$} (RP3);
 \end{tikzpicture}
 \caption{Alles zusammen.}
 \label{fig: alles zusammen}
\end{figure}
Da $\tilde{p}$ und $\tilde{q}$ Homöomorphismen sind ist auch $f : \tilde{p} \circ \tilde{q}^{-1} : \R P^3 \to \SO(3)$ ein Homöomorphismus, und da das Diagramm kommutiert ist
\[
 f \circ q
 = \tilde{p} \circ \tilde{q}^{-1} \circ q
 = \tilde{p} \circ \pi
 = p.
\]
$f$ ist auch durch die Eigenschaft, dass $f \circ q = p$ bereits eindeutig bestimmt: Ist nämlich $g : \R P^3 \to \SO(3)$ eine stetige Abbildung mit $g \circ q = p$, so ist
\[
 g(q(x)) = p(x) = f(q(x)) \text{ für alle } x \in D^3,
\]
da $q$ surjektiv ist also bereits $g(x) = f(x)$ für alle $x \in \R P^3$ und somit $f = g$.
































\end{document}
